\section*{Abstract}
\addcontentsline{toc}{section}{Abstract}

The Hygge programming language\cite{lecture_notes} is a didactic functional programming language designed by A. Scalas
for use in the course ``02247 - Compiler Construction''\cite{curriculum_02247} at the Technical University of Denmark (DTU).
A reference compiler, called \texttt{hyggec}\cite{hyggec}, is written in the \texttt{F\#} programming language and
generates assembly for the RISC-V instruction set architecture as its target language. This presents a challenge,
as DTU admits many international students to the MSc programmin in Computer Science and Engineering and only some of them
are familiar with functional programming and RISC-V. In an attempt to increase accessibility and ease the learning curve for students
without said experience, this paper presents a redesign of \texttt{hyggec} written in the \texttt{Java} programming language
called \texttt{JHygge}. Aside from being written in \texttt{Java}, this new compiler will also generate bytecode for
the Java Virtual Machine (JVM) as its target language instead of RISC-V, as incoming students are required to have
experience with \texttt{Java} and thus should have some level of familiarity with the JVM. Yet, the functional design
of \texttt{hyggec} has many properties that are desirable to retain. At the time of writing, the latest release of \texttt{Java},
version 24, offers improved support for functional programming, so this thesis will investigate relevant new features
and apply them to \texttt{JHygge}. Another new feature of \texttt{Java} 24 is the Class-File API\cite{jep484}, which
allows the programmer to manipulate \texttt{.class}-files will also be used in \texttt{JHygge} to generate bytecode
for the JVM. After having specified functional and non-functional requirements, proposed a redesigned architecture,
implemented and tested each of the compilation phases and Hygge language features, the final implementation of \texttt{JHygge}
showed the ability to generate JVM bytecode correctly implementing the semantics of all the Hygge language features
that should be supported using the Class-file API. Although having conducted an end-user test with ca. 200 current and
former students of the course ``02247'', the attempts to improve flexibility, extensibility and user-friendliness with
the new redesign could not be assessed as none of the invited students chose to participate.


