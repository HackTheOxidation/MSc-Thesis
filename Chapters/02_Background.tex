\chapter{Background and Prerequisite Knowledge} \label{sec:background}

In order to get the most out of the thesis, we recommend that the reader is familiar with a quite wide variety of topics.
In this section, we will go through all of required background and prerequisite knowledge, either in great detail in the case of new insights discovered
as part of the thesis project work, or at a glance if other sources already provide an appropriate in-depth explanation of well-established knowledge.

The knowledge used and discovered in this thesis project spans from rigid math-based topics such as programming language theory to softer subjects such 
programming paradigms, design patterns and compiler architecture with a very concrete and pragmatic approach by being specifically centered around the Java 
ecosystem.

\section{Programming Paradigms and Styles}

Programming languages have evolved substantially since the first programming language came 
to fruition with FORTRAN in 1957. Yet, the operating principles of the underlying hardware 
haven't changed nearly as much. 

Many newer language features provide abstractions and 
programming constructs that have no effect on how the computer processes data, such as
classes, namespaces and modules, but these are still important to programmers as it allows
the programmer to divide the software into logical and coherent components for increased
readability, reduced maintenance and re-usability. Paradigms like Structural programming, 
Procedural Programming and Object-Oriented Programming are examples of ways to associate 
state and behaviour of programs. Other programming features are concerned with how 
computations are performed. 

Most modern processors are built on the fundamental principles of the abstract Turing 
machine, which is a type of state machine. A state machine changes its state, which is the 
essence of Imperative programming; the programmer instructs the computer with commands on
how and when to mutate state. At the other end of the spectrum, Functional programming takes 
a different approach by effectively banning mutability and isolating pure computations from 
side-effects. There is no objectively better way to declare one paradigm as superior to the 
others. Rather, some paradigms are more suitable for certain types of applications and 
problems. 

In the following subsections, we will briefly review the core principles of the two most
prevalent programming paradigms at the time of writing. That is, Object-Oriented Programming
and Functional Programming.

\subsection{Object-Oriented Programming and Design}

The term Object-oriented programming was first coined by Alan Kay with the introduction of
the SmallTalk language, yet other languages like Simula already had support for OOP features
like classes, inheritance and run-time polymorphism even though the OOP term wasn't commonly
used yet. The more commonly used or modern interpretation of OOP came forth in the 1990's
where languages like Java sprung into existence. We shall focus on the latter interpretation
as this is the most popular one today due to the fact that the most popular languages,
such as Java, Python and JavaScript/TypeScript, that describes themselves as Object-Oriented
implements OOP in this way.

OOP has the following four core principles (sometimes referred to as the four pillars) that it adheres to:

\begin{enumerate}
\item Abstraction
\item Encapsulation
\item Inheritance
\item Polymorphism
\end{enumerate}

While one might argue that some of these principles are not exclusive to OOP, Inheritance is perhaps one of
the most distinguishable features that make OOP stand out from other paradigms.

\subsection{Functional Programming}

Functional programming has its root in Alonso Churchs Lambda Calculus, which was shown to
be computationally equivalent to a Turing Machine; everything a Turing Machine can compute, 
Lambda Calculus can do the same and vice versa. The defining characteristic of Lambda 
Calculus is that the fundamental unit of abstraction is anonymous functions, or lambdas. 

\section{Java and the JVM ecosystem}

In this section, we introduce the relevant material regarding the Java platform

\subsection{Modern Java (JDK 21+)}

When Java was initially released in 1995, it mainly had support for Object-oriented 
programming, which was quite popular at the time, with features like classes, objects, 
inheritance in the form of both abstract classes and interfaces and method overloading.
Since Java 8, there has been a shift towards providing support for more Functional 
programming features like anonymous functions (lambdas), higher-order functions, immutable
structures (records), pattern matching and discriminated unions in the form of sealed 
interfaces. We will briefly review a select number of these newer functional features
and compare them to other functional languages.

\subsection{The Java Virtual Machine (JVM) and JVM bytecode}

The Java platform is more than the Java language. It consists of its own virtual machine,
the Java Virtual Machine (JVM), which can execute ``class''-files with JVM bytecode, among other things.
The architecture of the JVM is based on a stack machine, where every frame is pushed on and popped from
a stack to perform computations. There are multiple different stacks at play in the JVM.

According to the Java SE 24 ``Java Virtual Machine Specification'', the JVM is an abstract computing machine
and does not assume anything about the underlying hardware platform that it runs on.

\section{Programming language theory}

\subsection{Context-Free Grammars}

\subsection{Parsers}

\subsection{Operational Semantics}

\subsection{Type systems}

\section{Compiler Construction}

\subsection{Architecture and Design}
