\chapter{Introduction}

Today, compilers, interpreters, higher-level languages and many other forms of programming language technologies are found
in all sorts of software toolchains and applications. From compilers and build systems to embedded interpreters in user
applications, programming language technologies have become ubiquitous in the computing landscape. Due to this fact, it is
more important than ever to teach students, aspiring programmers and software engineers how they themselves
can create their own compilers from the ground up. That is exactly the purpose of the course ``02247 - Compiler Construction''\cite{curriculum_02247}
at the Technical University of Denmark (DTU), where graduate students are taught the basics of how to build a programming language.
The course prerequisites assumes that the students are familiar with basic programming, data structures and algorithms and
the fundamentals of the theory of computation (mostly inference rules, regular expressions and context-free grammars), but
other than that there are no further requirements.

In the course, the students goes through a project, where they have to design and implement iterations of the Hygge programming language.
The Hygge programming language is a toy language designed by A. Scalas for didactic purposes. A the beginning of the course,
the students can either choose to be adventurous and build everything from scratch, using their programming language of choice,
or to follow a more standardized route by receiving a basic compiler, called \texttt{hyggec}\cite{hyggec}, and extending it with new
language features as the course progresses.

This basic compiler, \texttt{hyggec}, which we may also refer to as ``the reference implementation'', is written in \texttt{F\#},
an ML-based functional programming language, and the Hygge language employs a design that makes use of some functional programming concepts.
Here is the issue: far from every student has experience with a functional programming language and even fewer with \texttt{F\#}
in particular. This means that the student has an even steeper learning curve having to learn a new language and paradigm
in addition to the compiler construction curriculum. Also, \texttt{hyggec} emits RISC-V assembly as its target language\cite{lecture_notes},
which the student may not have been exposed to either.

The official programming language of DTU Compute, the Department of Applied Mathematics and Computer Science at DTU, is Java, so
all incoming students for the Master's programme in Computer Science and Engineering (CSE) are expected to have experience with it.
So too is a course in algorithms and data structures, as the students should be familiar with basic data structures like stacks, trees and lists.

Given this knowledge, a way to improve the availability of the course 02247 to more students would be to write a new Hygge compiler in Java. 
This would get rid of having to learn a new language to write the compiler, but the challenge of learning assembly and machine architecture remains,
so what would be a suitable target language to replace RISC-V? The students already know Java and thus have some experience working with 
the Java Virtual Machine (JVM). The JVM is stack-based and sits at a higher level of abstraction compared to assembly\cite{jvm_spec}, so it should be much simpler 
to learn for someone with no exposure to machine languages and embedded development. The last issue is that the functional programming features
of \texttt{F\#} allows for a concise, succinct and extensible implementation of the compiler, which would be desirable traits to retain. 
Although being know as an Object-Oriented language, the later releases of Java since version 21 has added a number of functional programming features.

As mentioned, the JVM is a stack-based virtual machine\cite{jvm_spec}, but what does that mean for code generation? Is is possible to implement all of the semantics
and the type system such that it will run on the JVM just like in RISC-V? Or will the nature of the JVMs design impose restrictions on the Hygge language?
Just like code generation is the emphasis of the course ``02247 - Compiler Construction''\cite{curriculum_02247}, so will it be the emphasis for this Master's Thesis project.
To summarize, this project has two goals. Firstly, explore the latest versions of Java (JDK 21 and newer) with an emphasis on Functional Programming features and then re-architect, redesign and rewrite the reference implementation in Java for an extensible, flexible and user-friendly compiler.
Secondly, explore the JVM and its bytecode target language and attempt to implement as many features of the Hygge programming language as possible on the JVM.
