\chapter{Requirements and Goals} \label{sec:requirements}

In this section we state and explain the requirement and the goals of this thesis project as well as some non-goals to help clarify the scope
of the project.

We distinguish between requirements and goals. A requirement specifies features, functionality and properties of the compiler to be written.
This is a classic systems engineering approach to describe the things that a system can do (functional requirements) and adding constraints as
to how it should do the things (non-functional requirements). This is very product-centric.

A goal on the other hand is about obtaining an answer a research question. That is, the success of the project does depend on whether the research
question yields a desirable answer, but only that more insights are gained about the problem that the question is concerned with. In other words,
we seek to get closer to an answer or gain a more refined perspective.

\section{Requirements}

In this section, we use the words ``must'', ``should'', ``could'' and ``won't'' in accordance with their meaning defined by the MoSCoW prioritization
technique. To briefly explain, ``must'' mean that the requirement is critical for the success of the project and thus must be achieved. The
word ``should'' means that although the requirement is not critical its completion will bring a lot of value to the project, which could impact
the result, but is not essential. ``Could'' means that there is some value to be gained by fulfilling the requirement, but far less than ``should''
and is entirely optional.
Finally, ``won't'' is the same as declaring a requirement as ``could'', yet there is no way that it is doable within the allocated time frame for the project.
As a final note, the set of ``must''-requirements form the Minimum-Viable-Product (MVP) in the sense that the project won't be a success unless these
requirements are met by the final implementation of the new Hygge compiler, which should henceforth be known by the name \texttt{JHygge}.

The requirements specification is informal in the sense that requirements are specified using natural language rather than rigid unambiguious mathematics.

For the requirements, we must write a compiler for the Hygge programming language, which as a minimum supports the Hygge0 version of the language,
but should support all the features of the reference implementation, \texttt{hyggec}.
To validate and verify the implementation of the Hygge programming language, \texttt{JHygge} must include an extensive and rigid test suite.

To summarize, the \texttt{JHygge} compiler has the following functional requirements:

\begin{itemize}
  \item \texttt{JHygge} must implement all of the Hygge 0 language specification.
  \item \texttt{JHygge} should implement mutable variables.
  \item \texttt{JHygge} should implement ``while-do'' loops.
  \item \texttt{JHygge} should implement functions.
  \item \texttt{JHygge} should implement structures and field selections.
  \item \texttt{JHygge} should implement discriminated unions and recursive types.
  \item \texttt{JHygge} could implement peephole optimization.
  \item \texttt{JHygge} could implement ANF optimization.
  \item \texttt{JHygge} won't implement Register Allocation.
  \item \texttt{JHygge} won't implement Closures.
  \item \texttt{JHygge} should implement a CLI interface akin to that of \texttt{hyggec}.
\end{itemize}

With the use of the word ``implement'' in the requirements, we mean that the implementation will provide the same level of support
for the given feature as the reference compiler \texttt{hyggec} without student modifications. For example, discriminated unions
and recursive types are only supported by the typechecker and the interpreter, but not the code generator, whereas all the phases
are supported for functions.

Additionally, the \texttt{JHygge} compiler has the following non-functional requirements:

\begin{itemize}
  \item \texttt{JHygge} must be written in the Java programming language version 24 and make use of functional programming features and design.
  \item \texttt{JHygge} must have a test suite that can run unit and integration tests for the compile pipeline stages.
  \item \texttt{JHygge} must make use of a parser generator for implementing the parser, for example ANTLR4, JavaCC, etc.
  \item \texttt{JHygge} must make use of a framework for generating JVM bytecode, for example ASM, Class-file API, ByteBuddy, etc.
  \item \texttt{JHygge} should make use of a build and dependency management system, for example Maven, Gradle, Bazel, etc.
\end{itemize}

Together, these fifteen requirements form the specification for the \texttt{JHygge} compiler.

\section{Goals}

There are two primary goals of this MSc thesis project:

\begin{itemize}
  \item To explore the functional language features of newer Java releases and re-architect, redesign and rewrite \texttt{hyggec} in the form of \texttt{JHygge}.
  \item To explore the Java Virtual Machine along with its bytecode target language and attempt to implement as many features of the Hygge programming language as possible on the JVM.
\end{itemize}

\section{Non-goals}

To create a successor language to \texttt{Hygge} or vastly alter its specification is not a goal of the project. Rather, we want to create a
``playground'' framework of sorts that allows the students to easily extend the \texttt{Hygge} language as part of the course work in 02247. 
It is also not a goal to implement a parser or parser generator, as many well-tested and proven tools already exist as well as the fact that
parsing is not an emphasis in neither the course ``02247 - Compiler Construction'' nor in this thesis. Although code generation for the JVM platform
is very much the emphasis of this thesis, it is not a goal to create a framework to read, write and update Java class-files, as these also exists 
(most notably ASM and the Class-File API from Java 24).
