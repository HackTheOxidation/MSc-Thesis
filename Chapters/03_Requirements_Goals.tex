\chapter{Requirements and Goals} \label{sec:requirements}

In this section we state and explain the requirement and the goals of this thesis project as well as some non-goals to help clarify the scope
of the project.

We distinguish between requirements and goals. A requirement specifies features, functionality and properties of the compiler to be written.
This is a classic systems engineering approach to describing the things that a system can do (functional requirements) and adding constraints as
to how it should do the things (non-functional requirements). This is very product-centric.

A goal on the other hand is about obtaining an answer a research question. That is, the success of the project does depend on whether the research
question yields a desirable answer, but only that more insights are gained about the problem that the question is concerned with.

\section{Requirements}

In this section, we use the words "must", "should", "could" and "won't" in accordance with their meaning defined by the MoSCoW prioritization
technique. To briefly explain, "must" mean that the requirement is critical for the success of the project and thus must be achieved. The
word "should" means that although the requirement is not critical its completion will bring a lot of value to the project, which could impact
the result, but is not essential. "Could" means that there is some value to be gained by fulfilling the requirement, but far less than "should"
and is entirely optional.
Finally, "won't" is the same as declaring a requirement as "could", yet there is no way that it is doable within the allocated time frame for the project.
As a final note, the set of "must"-requirements form the Minimum-Viable-Product (MVP) in the sense that the project won't be a success unless these
requirements are met by the final implementation of the new Hygge compiler, which should henceforth be known by the name \texttt{JHygge}.

For the requirements, we must write a compiler for the Hygge programming language, which as a minimum supports the Hygge0 version of the language,
but should support all the features of the reference implementation, \texttt{hyggec}.
To validate and verify the implementation of the Hygge programming language, \texttt{JHygge} must include an extensive and rigid test suite.

\section{Goals}

\section{Non-goals}

To create a successor language to \texttt{Hygge} or vastly alter its specification is not a goal of the project. Rather, we want to create a
"playground" framework of sorts that allows the students to easily extend the \texttt{Hygge} language as part of the course work in 02247. 
