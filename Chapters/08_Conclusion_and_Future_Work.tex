\chapter{Conclusion}

To summarise, in this MSc Thesis project we have gone through the phases of defining requirements, architecting, designing, implementing
and testing a new compiler for the Hygge programming language called \texttt{JHygge}, which is written in the \texttt{Java} programming
language (JDK 24) and generates bytecode for the JVM as its target. Analysing the existing reference compiler, \texttt{hyggec}, written in
\texttt{F\#}, generating RISC-V assembly, also revealed opportunities for improvement with regards to the extensibility and flexibility
of the design. This analysis allowed the authors to define 11 functional requirements (Figure \ref{fig:functional_requirements}) and
5 non-functional requirements (Figure \ref{fig:nonfunctional_requirements}) as well as state two goals, or research questions, for
the \texttt{JHygge} project centered around functional programming and design in ``modern'' \texttt{Java} (\textbf{G01}), and code generation
for the Java Virtual Machine (\textbf{G02}).
