\chapter{Conclusion}

To summarise, in this MSc Thesis project we have gone through the phases of defining requirements, architecting, designing, implementing
and testing a new compiler for the Hygge programming language called \texttt{JHygge} for use in the course ``02247 - Compiler Construction''\cite{curriculum_02247}
at the Technical University of Denmark (DTU), which is written in the \texttt{Java} programming
language (JDK 24) and generates bytecode for the JVM as its target. Analysing the existing reference compiler, \texttt{hyggec}, written in
\texttt{F\#}, generating RISC-V assembly, also revealed opportunities for improvement with regards to the extensibility and flexibility
of the design. This analysis allowed the authors to define 11 functional requirements (Figure \ref{fig:functional_requirements}) and
5 non-functional requirements (Figure \ref{fig:nonfunctional_requirements}) as well as state two goals, or research questions, for
the \texttt{JHygge} project centered around functional programming and design in ``modern'' \texttt{Java} (\textbf{G01}), and code generation
for the Hygge language on the Java Virtual Machine (\textbf{G02}). With requirements and goals set, a new architecture and design has been proposed
based on an intersection of principles from the Object-Oriented and Functional programming paradigms. Aside from the requirement to use \texttt{Java}
release version 24, in this architecture, it was also decided to use \textit{Gradle}\cite{gradle} for build and dependency management, \texttt{ANTLR4}\cite{antlr4}
for parser generation, and the Class-file API\cite{jep484} to manipulate \texttt{.class}-files to generate JVM bytecode. For testing, an automated
testing suite akin to that of \texttt{hyggec}\cite{hyggec} was developed, which tests each of the compilation phases with user-defined Hygge test
programs organized in a directory structure. Furthermore, an end-user test was conducted, where current and former students of the course
``02247 - Compiler Construction'' were invited to text the extensibility and user-friendliness of \texttt{JHygge} by extending it with an
expression for arithmetic subtraction. Unfortunately, none of the ca. 200 invited students chose to participate in the test, so there is no data 
to assess whether or not the newly designed \texttt{JHygge} achieves better flexibility, extensibility and user-friendliness.
However, it was possible to assess the degree to which \texttt{JHygge} achieves its functional and non-functional requirements;
the final implementation of \texttt{JHygge} was deemed to fulfil 9 out of 11 functional requirements and all 5 non-functional requirements.
The two functional requirements not met by \texttt{JHygge}, \textbf{F07} and \textbf{F08} regarding ``peephole'' optimization
and Administrative-Normal-Form (ANF), respectively, were also prioritized low and as optional extras not essential to the success of the project
as indicated by the initial MoSCoW\cite{moscow} analysis.
Regarding the goals of the project, it was found that support for Functional programming in \texttt{Java} (\textbf{G01}) has
indeed improved in the recent releases, yet there are still a lot of room for improvement as some of the more advanced functional programming
features are poorly supported or missing entirely. On a more positive note, bytecode generation for the Hygge programming language on the
JVM (\textbf{G02}) turned out to be a success using the new Class-file API from \texttt{Java} 24; it was possible to generate JVM bytecode
for every language feature of Hygge in declarative style, and even emulate the structural type system of Hygge on the nominally-typed JVM.
