\chapter{Evaluation}

In this chapter, we discuss the findings and the lessons learned during this MSc Thesis project. We evaluate the final implementation
of the \texttt{JHygge} compiler based on the requirements stated in the \ref{sec:requirements} chapter and the results achieved in
\ref{sec:testing}. Finally, we discuss the goals of this MSc Thesis project and the degree to which they have been achieved.

\section{Does JHygge fulfil the requirements?}

In this section, we discuss the degree to which the new \texttt{JHygge} compiler fulfils the functional and non-functional requirements
stated in figures \ref{fig:functional_requirements} and \ref{fig:nonfunctional_requirements}. In table \ref{table:requirements_fulfillment}, we provide an overview of Hygge language feature support for each of the four compilation phases of
\texttt{JHygge} along with a summary of whether the level of support for a given feature matches that of \texttt{hyggec}.

\begin{table}[H]
\centering
\begin{tabular}{lccccc}
\multicolumn{1}{l|}{}       & \multicolumn{1}{l}{Parsing} & \multicolumn{1}{l}{Typechecker} & \multicolumn{1}{l}{Interpreter} & \multicolumn{1}{l}{Codegen} & \multicolumn{1}{l}{\texttt{hyggec} compatible?} \\ \cline{1-1}
Primitive values             & yes                                & yes                              & yes                                & yes                                & yes                                                              \\
Variables                    & yes                                & yes                              & yes                                & yes                                & yes                                                              \\
Basic Arithmetic  & yes                                & yes                              & yes                                & yes                                & yes                                                              \\
Basic Logic   & yes                                & yes                              & yes                                & yes                                & yes                                                              \\
Typing                       & yes                                & yes                              & yes                                & yes                                & yes                                                              \\
Let-bindings                 & yes                                & yes                              & yes                                & yes                                & yes                                                              \\
Assignments                  & yes                                & yes                              & yes                                & yes                                & yes                                                              \\
Assertions                   & yes                                & yes                              & yes                                & yes                                & yes                                                              \\
Read and Print I/O           & yes                                & yes                              & yes                                & yes                                & yes                                                              \\
If-Then-Else                 & yes                                & yes                              & yes                                & yes                                & yes                                                              \\
Mutable                      & yes                                & yes                              & yes                                & yes                                & yes                                                              \\
While do                     & yes                                & yes                              & yes                                & yes                                & yes                                                              \\
Functions                    & yes                                & yes                              & yes                                & yes                                & yes                                                              \\  
Structs and Fields & yes                                & yes                              & yes                                & yes           & yes                                         \\ 
Unions and \texttt{match}  & yes                                & yes                              & yes                                & no                                 & yes                                                             
\end{tabular}
\caption{Overview of feature support for each of the compilation stages along with a summarizing column showing compatibility with \texttt{hyggec}.}
\label{table:requirements_fulfillment}
\end{table}

Note that although code generation is not supported for unions and pattern matching as shown in Table \ref{table:requirements_fulfillment}, \texttt{JHygge} is still compatible with \texttt{hyggec}
as this is a project option for the students of the course ``02247 - Compiler Construction'' and therefore not supported out-of-box for \texttt{hyggec}
either. Aside from that, the table shows that \texttt{JHygge} implements the same level of support for each feature as \texttt{hyggec}.
When comparing this result with the functional requirements stated in figure \ref{fig:functional_requirements}, this shows fulfilment of all requirements related to support for features of the Hygge language. That is, functional requirements \textbf{F01} to \textbf{F06}. \textbf{F09} and \textbf{F10} are also met as \texttt{JHygge}
does not support neither closures nor register allocation.

Administrative-Normal-Form (ANF), \textbf{F08}, and ``peephole'' optimization, \textbf{F07}, have not been implemented, but they were also prioritised as low and as optional
extras not essential to the success of \texttt{JHygge}. Besides, one of the purposes of ANF in \texttt{hyggec} is to optimize register
allocation for the RISC-V, where there is a limited number of registers available (Module 11 of \cite{lecture_notes}), but since the JVM doesn't have the concept of registers
this partially justifies assigning it to a lower priority. As for optimization in general, while it is part of the course
curriculum of ``02247 - Compiler Construction''\cite{curriculum_02247}, one could say that this may not be as beneficial, since the JVM utilizes a Just-In-Time (JIT)
compiler for optimizations at run-time\cite{jvm_spec}, so some attempts to optimize at compile-time may be in vain. Optimizations like constant folding
and constant propagation would still be useful, however. As stated, register allocation has not been implemented as the JVM does not have
this concept, so this aligns with the functional requirements.

Regarding the last of the functional requirements, \textbf{F11}, we can also affirm that a CLI akin to that of \texttt{hyggec} has been implemented. As
mentioned in the chapter on testing \ref{sec:testing}, this CLI proved useful when testing the various compilation phases.
As such, we have asserted that this requirement for a CLI has been fulfilled by the \texttt{JHygge} compiler.

As for the non-functional requirements stated in figure \ref{fig:nonfunctional_requirements}, we can conclude that \texttt{JHygge} is indeed written in the \texttt{Java} programming language and does make use of functional programming features and design,
thus fulfilling \textbf{NF1}.
An evaluation as to, how well \texttt{Java} 24 supports functional programming
is discussed later in this chapter. From the \ref{sec:testing}, we saw the use of the automated testing suite, so \textbf{NF2}
has also been covered. During the design stage described the \ref{sec:architecture}, it was already chosen to make use
of ANTLR4 for parsing (\textbf{NF3}), Gradle for build and dependency management (\textbf{NF5}), and
the \texttt{Java} 24 Class-file API\cite{jep484} for JVM bytecode generation (\textbf{NF4}) at this point.
Hence, we can conclude that the \texttt{JHygge} compiler meets all the non-functional requirements.

\section{What about the Goals for the \texttt{JHygge} project?}

In this section, we reflect on the two goals, or research questions, of the
\texttt{JHygge} project, \textbf{G01} and \textbf{G02}, to see if we have gained
any new knowledge and have come closer to answering these questions.

\subsection{\textbf{G01}: Modern Java is not quite functional, yet}

In this thesis, we have investigated the newer versions of \texttt{Java} specifically with an emphasis on functional programming features.
Although, some of the functional features have been added to the language, \texttt{Java} is still primarily an Object-Oriented language.
It lacks a lot of the more powerful features of other functional languages such as \texttt{Haskell}, \texttt{F\#} and \texttt{Rust},
most notably, functions as first-class citizens, support for monads and functors, algebraic data types.

Regarding monads, the \texttt{Java} standard library does provide the \texttt{Optional}\cite{java_optional} monad, but there are a few problems such as the
fact that it is not implemented as a discriminated union (sealed interface) and thus is not supported by pattern matching in
\texttt{switch}-expressions. Also, instances of \texttt{Optional} are not type-safe as one can still assign \texttt{null} to them.

Another very useful and common monad, which is also used in \texttt{hyggec}, is the \texttt{Result}-monad. This is useful for monadic
and explicit error handling in functional languages, but it is simply lacking in \texttt{Java} 24. Instead, \texttt{Java} relies on
exceptions as its primary way of handling errors. As such, the \texttt{JHygge} implementation relies on exceptions for error handling.
When comparing the implementation of \texttt{JHygge} to that of \texttt{hyggec}, \texttt{hyggec} does make reasonable use of the \texttt{Result}
monad as it is properly supported in \texttt{F\#}. This would have made implementing the typechecking easier. Instead, due to the lack
of \texttt{Result}, we chose to use exceptions instead for the typechecking and (JVM) code-generation phases. Exceptions do produce a
side-effect, which isn't desirable, but it does make the code simpler to understand for students who primarily have experience using
\texttt{Java} in the more common OOP style. As a final note on the \texttt{Result} monad, while it is not the goal of this thesis to attempt
to ``complete'' support for functional programming features in \texttt{Java}, it is baffling to see that this type has not been implemented
in \texttt{Java} when \texttt{Optional} has. As a minor experiment, a proposed implementation for discriminated union-based \texttt{Result}
monad has been made and it can be found in Appendix \ref{appendix:result_java}.

Furthermore, \texttt{Java} also lacks support for tuples as a built-in data type; this means that one is forced to create small ad-hoc
classes for grouping related data in situations where one does not care about naming a new type. One example of this is \texttt{InterpreterResult},
which only serves the purpose of storing a \texttt{RuntimeEnvironment} and a \texttt{Node}. The same goes for \texttt{LeftRightReductionResult}.
Both of these are single use, single purpose construction whose names are not important.

\subsubsection{The redesign of a Hygge compiler}

Regarding the architecture and design part of the goal, it becomes difficult toobtain a clear-cut and definitive answer.
With the new architecture and design for \texttt{JHygge}, we have attempted to improve extensibility and flexibility foremost by combining design patterns
and principles from both Functional and Object-Oriented programming. However, whether the new design achieves the promises of greater extensibility
and flexibility remains to be seen. These architectural and design properties can only be tested qualitatively by its intended users. Unfortunately, although
an end-user test has been planned, none of the students invited chose to participate. Therefore, we cannot say nor conclude anything about these properties.

\subsection{\textbf{G02}: The Class-file API in action}

Although the previous section evaluating our experiences with \texttt{Java} as a functional programming language came out as rather disappointing,
this section will be a lot more positive as we find that the Java 24 Class-file API\cite{jep484} to not just work as advertised, but also to provide a
reasonably intuitive and declarative API. As shown throughout the \ref{sec:codegen},
it is indeed possible to achieve JVM bytecode generation for the Hygge language semantics. With the implementation of \texttt{JHygge},
as specified in table \ref{table:requirements_fulfillment}, JVM bytecode generation has been implemented for all Hygge language features, where the reference compiler, \texttt{hyggec},
also supports code generation for RISC-V.

The question on whether it is easier for the programmer to work with code generation for the JVM
rather than RISC-V is a subjective one. As none of the invited students participated in the end-user test, we cannot provide any statistical insight
into the preferred target language for the intended audience. However, it is the authors own opinion and experience that working with JVM bytecode as
a target language is easier than working with RISC-V for a few reasons. The primary reason is the additional abstractions, such as methods and classes,
that are built in to the JVM from its inception. Automatic memory management in the form of Garbage Collection also relieves the programmer of having
to remember to free all heap-allocated manually, which is especially useful when working with Hygge \texttt{struct}s. Finally, the authors find that
the stack machine of the JVM is easier to comprehend as a mental model than the RISC-V instruction set architecture; there are no registers to keep
track of, only an array for local storage, which can store any type, be it either a primitive or a reference type.

Since code generation for \texttt{union}s and pattern matching is not supported by \texttt{hyggec} as this is one of the project ideas for the students
to do and thus it has not been implemented in \texttt{JHygge}. However, if one were to provide a few hints for a prospective student on how to implement
\texttt{union}s and pattern matching in \texttt{JHygge}, it would be an idea to exploit the new \texttt{sealed interface}\cite{jep409}- and \texttt{record}\cite{jep395}-features
as the Hygge syntax requires that a \texttt{union} is declared as a type with a name and all labels/variant before use. In turn, this would enable
the use of \texttt{switch}-expressions to enforce exhaustive patterns. Another feature for a student project would be to implement closures;
from the description of the project idea in the lecture notes\cite{closures}, there are different closure ``styles'' that can be implemented.
In this case, a hint for the student could be to make all variables and \texttt{let}-bindings public attributes of the parent class \texttt{HyggeMain}.
In the current implementation of \texttt{JHygge}, the value of a variable is simply stored in the local storage array and is thus only accessible in
the current stack frame.

\subsubsection{Structural typing on the nominal JVM}

During the implementation of JVM bytecode generation for \texttt{struct}s and field selections, an interesting problem arose.
As per the Hygge language specification, the typing system for Hygge is structural. With \texttt{hyggec} it has been shown
that implementing this typing system in RISC-V is possible. The story for the JVM back-end in \texttt{JHygge} is somewhat different;
we found that implementing the structural Hygge typing system on the JVM requires some additional layer of abstraction, as
the JVM primarily enforces its own nominal sub-typing rules based on inheritance at the bytecode level.
However, the solution turned out to be simple as \texttt{HashMap<String, Object>}\cite{hashmap} provided a suitable abstraction.

In the \texttt{Scala} programming language, it is also possible to imitate structural typing using the trait \texttt{scala.Selectable}\cite{scala3}, but this is
due to the additional abstractions provided by the \texttt{Scala} standard library and does not change anything about the nominal
type system at the bytecode level of the JVM.

