\chapter{Implementation}

At this point, we have arrived at a suitable architecture and design for \texttt{JHygge}, the new Hygge compiler, so this chapter concerns the actual
implementation. Here, we emphasize the use of the new functional features of modern Java (JDK 21 and up) and present the highlights
of the final implementation of \texttt{JHygge}. 

\section{Use of Functional Java: Records, Sealed Interfaces, Pattern matching and Monads}

As the purpose of this thesis is also to explore the newer functional features of \texttt{Java}. The existing implementation of \texttt{hyggec}
is indeed very functional, so we attempted to write the implementations of both the AST, typechecker and interpreter in such a way that
the \texttt{JHygge} versions are as close to those of \texttt{hyggec}, disregarding the architectural and design differences. Some of
the newer functional features of \texttt{Java} we wanted to investigate include records\cite{java_record}\cite{jep395} (immutable classes), sealed interfaces (discriminated unions)\cite{jep409},
switch expressions (pattern matching) and the introduction of some monads, most notably \texttt{Optional}. In the following sections,
we show concrete examples of implementation, where we have attempted to use said features and compare them to the corresponding code
in \texttt{hyggec}.

\subsection{Records and Sealed Interfaces for functional data modelling}

The \texttt{Node} and \texttt{Expr} data model for the AST from \texttt{hyggec} is one of the parts that we wish to carry over in \texttt{JHygge}.
This requires support for discriminated unions and recursive data type, the former of which was introduced in JDK 17 in the form of sealed
interfaces; one declares a finite number of exact specializations upon defining the interface itself. This gives the \texttt{Java} compiler
information about the possible specialization, which it can use to perform case analysis and thus enforce exhaustive pattern matching. Let's
see records and sealed interfaces in action in the form of \texttt{JHygge}s implementation of \texttt{Expr} in figure \ref{fig:expr_sealed_interface}.

\begin{figure}[H]
\centering 
\begin{lstlisting}[language=Java]
// From Expr.java, reformatted to be more compact and fit within the margins
public sealed interface Expr<E, T>
    permits Expr.UnitVal, Expr.BoolVal, Expr.IntVal, Expr.FloatVal,
        Expr.StringVal, Expr.Var, Expr.BinaryOperator, Expr.Not,
        Expr.ReadInt, Expr.ReadFloat, Expr.Print, Expr.PrintLn,
        Expr.If, Expr.Seq, Expr.Type, Expr.Ascription,
        Expr.Assertion, Expr.Let, Expr.Assign, Expr.While,
        Expr.Lambda, Expr.Application, Expr.StructCons,
        Expr.FieldSelect, Expr.Pointer, Expr.UnionCons,
        Expr.Match {
    // ...
}
\end{lstlisting}
\caption{The implementation of Hygge expressions as a \texttt{sealed interface} specifying the exact allowed specializations.}
\label{fig:expr_sealed_interface}
\end{figure}

Similar to discriminated unions in \texttt{F\#}, all cases are declared for the sealed interface \texttt{Expr}. However, unlike \texttt{F\#},
the cases have to be defined as separate classes, records or interfaces. In the case that a specialization is a class, they have to be \texttt{final}.
Let us examine an example of one of the mutually-recursive specializations of \texttt{Expr}, which also happens to be a \texttt{record}\cite{java_record}, namely the \texttt{Not}-expression for which the code is shown in figure \ref{fig:not_record}.

\begin{figure}[H]
\centering
\begin{lstlisting}[language=Java]
  // From Expr.java, Not<E, T> is an inner class of Expr<E, T>
  record Not<E, T>(Node<E, T> arg) implements Expr<E, T> {
    @Override
    public String prettyPrint(int indentation) {
      indentation += 2;
      return "Not(arg=" + arg.expr().prettyPrint(indentation) + ")";
    }
  }
\end{lstlisting}
\caption{An implementation of the Hygge \texttt{Not} expression in the form of a \texttt{record}.}
\label{fig:not_record}
\end{figure}

Declaring \texttt{Not<E, T>} as a \texttt{record} has the same effect as if it had been declared as a class with private \texttt{final}
attributes that inherit from the \texttt{java.lang.Record}, where each attribute only has a getter, but not a setter, along with implicit declarations
of some utility methods, such as the \texttt{Object.equals()}-, \texttt{Object.hashCode()}- and \texttt{Object.toString()}-method. Now, this example uses the
\texttt{record} declaration syntax, which does all of the things mentioned above to remove boilerplate to ease the burden of the programmer.

Another important thing to note about the immutability of Java records is that they are shallowly immutable, meaning that although that the record
attributes, also called components, cannot be altered, the attributes of the record components can be mutable. That is, this \texttt{Not<E, T>} record
has an \texttt{arg} component of type \texttt{Node<E, T>}, which is also a record, but had \texttt{arg} been declared as a class with mutable
attributes, the programmer could still change the value of those attributes eventhough \texttt{Not<E, T>} is a record.

A record is also \texttt{final} in the sense that extending record types through inheritance is prohibitted; it cannot be used as a base for
further specialization. This makes sense in the context of sealed interfaces and pattern matching, where it would otherwise be difficult
to perform case analysis and reason about pattern exhaustion had some variant in the discriminated union been a regular class that could
have had an unknown number of specializations.

\subsection{Switch expressions for case-by-case definitions and dispatching}

In the preceding section, we mentioned pattern matching when discussing the use of discriminated unions. In order to handle the different variants
of a discriminated union, pattern matching gives the programmer a way to not only branch based on the value of the active variant, but also on
the properties and the structure of the value (in some languages). With JEP XXX, the \texttt{switch}-expression was introduced in Java, which behaves
similarly to the \texttt{match}-expressions in languages like \texttt{F\#} and \texttt{Rust}. Unlike the existing \texttt{switch}-statement,
the newer \texttt{switch}-expression returns a value and one can use the \texttt{when}-clause to branch to multiple different cases for the
same variant in a discriminated union. This is useful in multiple compiler phases every time the \texttt{Expr}-type is involved. Take
the \texttt{reduce}-method as an example; in order to reduce each variant of the \texttt{Expr}-type, the \texttt{switch}-expression is used
to delegate the responsibility of expression reduction to the appropriate reducer as per our design proposal. An excerpt for the code
for the \texttt{reduce}-method from \texttt{StandardHyggeInterpreter} is shown on figure \ref{fig:switch_expression_values}:

\begin{figure}[H]
\centering
\begin{lstlisting}[language=Java]
  // From StandardHyggeInterpreter.java, slightly reformatted
  @Override
  public Optional<InterpreterResult<E, T>> reduce(RuntimeEnvironment<E, T> env, Node<E, T> node) {
    return switch (node.expr()) {
      case UnitVal<E, T> _,
          BoolVal<E, T> _,
          IntVal<E, T> _,
          FloatVal<E, T> _,
          StringVal<E, T> _,
          Pointer<E, T> _ ->
          Optional.empty();
      case Var<E, T> v when env.isMutable(v.name()) ->
          InterpreterResult.of(env, env.mutables().get(v.name()));
      case Var<E, T> _ -> Optional.empty();
      case BinaryOperator<E, T> v -> binaryOperatorReducer.reduce(this, env, node, v);
      case Not<E, T> v ->
          (v.arg().expr() instanceof BoolVal<E, T> b)
              ? InterpreterResult.of(env, node.with(b.not()))
              : reduce(env, v.arg())
                  .map(a -> new InterpreterResult<>(env, node.with(new Not<>(a.node()))));
      // Remaining cases redacted for conciseness...
    };
  }
\end{lstlisting}
\caption{A simple use of a \texttt{switch}-expression to branch based on the Hygge expression contained in a \texttt{Node}. Undiscussed cases have been removed for the sake of brevity.}
\label{fig:switch_expression_values}
\end{figure}

As we can see in the listing, the cases can group multiple variants to branch to the same single expression, as with
the value variants \texttt{UnitVal<E, T>}, etc., which is implicitly returned/yielded. For variants like \texttt{Var<E, T>}, there are
two different cases to handle, where the \texttt{when}-clause is used to disguish between these cases without having to resort to
additional constructs for control flow creating unwanted nesting. One might point out, that this essentially a functional programming
version of the Visitor pattern known from Object-oriented design. So, this is a case, where both FP and OOP style want to solve the
same problem, but do it in different yet still somewhat similar ways.

However, while pattern matching with \texttt{switch}-expressions can be useful, there is still limited support for it as one
can only pattern match on reference types (classes, interfaces, enums, etc.) as of Java 24 (a proposal for use with primitives is in preview with JEP XXX).
That is, one cannot easily do pattern matching on the structure or value of a type. Take the list data structure as an example;
in \texttt{F\#}, one can match not only against the type, but also the value and patterns of values, i.e. matching against \texttt{x :: xs}
would mean that there is at least one element in the list and maybe more. These kinds of patterns are not supported in Java 24, so working with lists and
map types are less straight-forward. Take the \texttt{Seq}-record for the sequencing expression as an example; it has a record component,
\texttt{nodes}, which is of type \texttt{List<Node<E, T>>}. There is a case during interpretation, where the interpreter should apply
different reductions for a sequence expression depending on the number of successive expressions, namely none, one and more than one.
While one could use a \texttt{switch}-expression with \texttt{when}-clauses, this would create deep and nested code impairing readability
due to indentation. Instead, it is simpler to branch on the size of the list using \texttt{if}-statements as shown in figure \ref{fig:seq_reduce_implementation},
which contains the \texttt{Java}-implementation of the reduction rules for the sequencing expression.

\begin{figure}[H]
\centering
\begin{lstlisting}[language=Java]
  // From StandardHyggeControlFlowReducer.java
  @Override
  public Optional<InterpreterResult<E, T>> reduce(
      HyggeInterpreter<E, T> interpreter,
      RuntimeEnvironment<E, T> env,
      Node<E, T> node,
      Seq<E, T> seqExpr) {
    var nodes = seqExpr.nodes();

    if (nodes.isEmpty()) {
      return InterpreterResult.of(env, node.with(new UnitVal<>()));
    }

    if (nodes.size() == 1) {
      var last = nodes.getFirst();
      return ASTUtility.isValue(last)
          ? InterpreterResult.of(env, node.with(last.expr()))
          : interpreter
              .reduce(env, last)
              .flatMap(
                  result -> InterpreterResult.of(result.env(), node.with(result.node().expr())));
    }

    var first = nodes.getFirst();
    var rest = nodes.subList(1, nodes.size());
    return !ASTUtility.isValue(first)
        ? interpreter
            .reduce(env, first)
            .flatMap(
                result ->
                    InterpreterResult.of(
                        result.env(),
                        node.with(
                            new Seq<>(
                                Stream.concat(Stream.of(result.node()), rest.stream()).toList()))))
        : InterpreterResult.of(env, node.with(new Seq<>(rest)));
  }
\end{lstlisting}
\caption{Implementation for the reduction rules of \texttt{Seq}-expressions without the use of a \texttt{switch}-expression}
\label{fig:seq_reduce_implementation}
\end{figure}

This implementation is not written in a style that is as functional as the \texttt{hyggec} counterpart and it seems that
this is one of the compromises of a language like \texttt{Java}, which was not designed to be functional from the beginning.
Also, notice that there is a need to create a local variable, \texttt{nodes}, to \texttt{seqExpr.nodes()} because pattern
matching isn't supported for method arguments as it is in \texttt{F\#}.

\subsection{\texttt{Optional} for monadic error handling}

The Option/Maybe and Result/Either types are ways to handle errors for effectful/impure code in a way that resembles pure code.
This is in contrast to exceptions, which is the de facto default approach to error handling in \texttt{Java}. The benefit
of using monadic error handling is that it becomes immediately obvious that a function can fail from its type signature.
Furthermore, the type also communicates the kind of errors that can occur. The Option/Maybe type indicates that the function
may produce something or nothing, and in the case of nothing, there is no information about why. This is useful in cases,
where when a failure occurs, the programmer should not be concerned about why it failed. On the other hand, the Result/Either
type is used when the programmer is interested in the cause of failure and want to act on it.

In \texttt{Java}, there is the \texttt{java.lang.Optional} type, which is an implementation of the Option/Maybe monad. It
supports the $>>=$ (pronounced: ``bind'') and \texttt{return} operations in the form of the \texttt{.flatMap()}-method
and \texttt{Optional.of()}, respectively. In \texttt{JHygge}, this has been used for the implementation of the interpreter
reduction rules based on the algorithm from \texttt{hyggec} as reduction is applied until it is impossible to reduce further.
At that point, the last successful reduced \texttt{Node} is returned. In figure \ref{fig:struct_reducer_monads}, we show an excerpt from
\texttt{StandardStructReducer.java}, which showcases the use of \texttt{Optional} in the \texttt{reduce()}-method.

\begin{figure}[H]
\centering  
\begin{lstlisting}[language=Java]
// From StandardStructReducer.java, all other methods have been redacted ...
public class StandardStructReducer<E, T> implements StructReducer<E, T> {

  @Override
  public Optional<InterpreterResult<E, T>> reduce(
      HyggeInterpreter<E, T> interpreter,
      RuntimeEnvironment<E, T> env,
      Node<E, T> node,
      FieldSelect<E, T> fieldSelectExpr) {

    if (fieldSelectExpr.target().expr() instanceof Expr.Pointer<E, T>(int address))
      return FunctionalUtilities.tryFind(env.ptrInfo(), address)
          .flatMap(
              fields -> {
                var offset = fields.indexOf(fieldSelectExpr.field());
                return InterpreterResult.of(env, env.heap().get(address + offset));
              });

    if (!ASTUtility.isValue(fieldSelectExpr.target()))
      return interpreter
          .reduce(env, fieldSelectExpr.target())
          .map(result -> result.with(new FieldSelect<>(result.node(), fieldSelectExpr.field())));

    return Optional.empty();
  }

}
\end{lstlisting}
\caption{Implementation of the reduction rules for Hygge \texttt{struct}s making use of the \texttt{Optional} monad.}
\label{fig:struct_reducer_monads}
\end{figure}

As shown in the listing above, the both the \texttt{.flatMap()} and \texttt{.map()} operations as well as \texttt{Optional.of()}
indirectly (this is hidden in \texttt{InterpreterResult.of()} and \texttt{FunctionalUtilities.tryFind()}). Notice how using
this technique emphasizes the ``happy path'' of the program as error propagation becomes implicit due to the functor and monad
operations \texttt{.map()} and \texttt{.flatMap()} allow the programmer to focus on the data transformations instead of error
handling.

\section{Parsing Hygge source files with ANTLR4}

As mentioned in the chapter on Architecture and Design, we have selected ANTLR4 as the parser generator of choice. For starters,
we have translated the \texttt{FsLexYacc} lexer and parser files, \texttt{Lexer.fsl} and \texttt{Parser.fsy}, to ANTLR4
\texttt{.g4}-format. In appendix \ref{appendix:hygge_grammar}, we show the complete \texttt{.g4} grammar file for the final version of \texttt{JHygge}.

Initially, this grammar file was syntactically more similar to the E-BNF for the \texttt{Hygge} specification as an attempt to
simplify the grammar file as much as possible. However, the \texttt{Hygge} grammar specification does not take operator
precedence into account, so it was later decided to factor out left-recursion and enforce operator precedence by separating
production rules similar to that of \texttt{Parser.fsy} in \texttt{hyggec}.

By default, \texttt{ANTLR4} generates both a lexer and a parser along with classes for implementing listeners and visitors.
One could implement the rest of \texttt{JHygge} using the \texttt{ANTLR4} AST representation, but this AST is intended to
be consumed or transformed by listeners and visitors, respecitively. Instead, we want a more functional model of an AST.
In order to breach the gap between the \texttt{ANTLR4} representation and our desired \texttt{Node} and \texttt{Expr} data model,
we decided to create a visitor, \texttt{AntlrHyggeNodeVisitor}, that provides this transformation. In the listing below, we
see an excerpt of \texttt{AntlrHyggeNodeVisitor}, namely the \texttt{visitAssertion}, which transforms a \texttt{HyggeParser.AssertionContext}
to a \texttt{Node} with an \texttt{Expr.Assertion}-expression, as well as the code for the \texttt{Expr.Assertion} record.
This implementation is shown in figure \ref{fig:antlr_node_visitor}.

\begin{figure}[H]
\centering 
\begin{lstlisting}[language=Java]
 
  // From AntlrHyggeNodeVisitor.java
  @Override
  public Node<Object, Object> visitAssertion(HyggeParser.AssertionContext ctx) {
    return new Node<>(
        new Expr.Assertion<>(ctx.simpleExpr().accept(this)),
        Position.from(ctx.start, ctx.stop),
        null,
        null);
  }

  // From Expr.java
  record Ascription<E, T>(PreTypeNode tpe, Node<E, T> node) implements Expr<E, T> {
    @Override
    public String prettyPrint(int indentation) {
      return "Ascription(node="
          + node.expr().prettyPrint(indentation + 2)
          + ", tpe="
          + tpe.preType().toString()
          + ")";
    }
  }

\end{lstlisting}
\caption{Combined excerpts from \texttt{AntlrHyggeNodeVisitor.java} and \texttt{Expr.java} showing the visitor and the definition for the \texttt{Ascription}-expression. }
\label{fig:antlr_node_visitor}
\end{figure}

The creation of \texttt{Node}- and \texttt{Expr}-objects is mostly straight forward even though they are mutually recursive.
To parse the argument for the assertion, one simply calls the \texttt{accept()}-method on the subexpression for the visitor
context, which will recursively traverse and transform the AST to \texttt{Node}-instances. Most of the expression types
follow a similar implementation style as the one for assertions. There are some of these tranformations that are more complicated,
like function declarations, structs, unions, etc., but it is the same case for declaring the equivalent parser for \texttt{hyggec}
using \texttt{FsLexYacc}.

To compare the two parser implementations in a more quantitative way, the parser in \texttt{JHygge}
takes up 159 lines (\texttt{hygge.g4}) and 440 lines (\texttt{AntlrHyggeNodeVisitor}) for a total of 599 lines, where the lexer and
parser for \texttt{hyggec} is 100 lines (\texttt{Lexer.fsl}) and 305 lines (\texttt{Parser.fsy}) totalling 405 lines. The Java
implementation takes up approx. $48\%$ more lines than the \texttt{F\#} version, where the \texttt{AntlrHyggeNodeVisitor} is
the largest contributor.

\section{Generating JVM bytecode with the Class-file API from Java 24} \label{sec:codegen}

The Class-file API\cite{jep484} promises a declarative approach to reading, modifying and writing \texttt{.class}-files. The main parts of the
API used in the implementation of the JVM backend in \texttt{JHygge} is the \texttt{ClassBuilder}\cite{java_classbuilder} and the \texttt{CodeBuilder}\cite{java_codebuilder}.
A \texttt{ClassBuilder} is intended for modifying classes at a higher-level, such as adding attributes/fields, constructors and methods,
without directly handling bytecode. A \texttt{CodeBuilder} is then used to construct the sequence of bytecode instructions in
a method or a constructor. Both the APIs for \texttt{ClassBuilder}s and \texttt{CodeBuilder}s make use of the builder pattern, which is one of the
classic creational OOP design patterns.

Let's walk through an example to show, how the Class-file API is used concretely in \texttt{JHygge}. The code listing on figure \ref{fig:codegen_main} shows
the implementation of the \texttt{generateCode()}-method for the \texttt{JVMCodeGenerator}-class. This is the minimal requirement
for implementing the \texttt{HyggeCodeGenerator}-interface. As we see in figure \ref{fig:codegen_main}, we initially generate an empty \texttt{.class}-file,
which is always \texttt{HyggeMain.class}, containing the \texttt{HyggeMain} class. The \texttt{ClassBuilder}-API is then used to
generate the static \texttt{main()}-method, which the JVM will look for and use as an entry-point. To generate the code for
the body of the \texttt{main()}-method, we use the \texttt{CodeBuilder}-API. Similar to the typechecker and the interpreter,
code generation is split across different classes and methods by grouping language features together, so the overloaded and private
\texttt{generateCode()}-method will then dispatch to the correct code generator method depending on the top-level node in the AST.
Note also, that the \texttt{JVMCodeGeneratorEnvironment}-class keeps a reference to the parent \texttt{ClassBuilder} for the
\texttt{HyggeMain}-class. This will come in handy later, when generating code for functions, for example.

\begin{figure}[H]
\centering 
\begin{lstlisting}[language=Java]
  // From JVMCodeGenerator.java
  @Override
  public void generateCode(Node<E, T> ast, String fileName) throws IOException {
    var filePath = Path.of("HyggeMain.class").toAbsolutePath();
    System.out.println("Compiling '" + fileName + "' to: " + filePath);
    ClassFile.of()
        .buildTo(
            filePath,
            ClassDesc.of("HyggeMain"),
            classBuilder -> {
              var codeGeneratorEnvironment = new JVMCodeGeneratorEnvironment(classBuilder);
              classBuilder.withMethodBody(
                  "main",
                  ofDescriptor("([Ljava/lang/String;)V"),
                  ClassFile.ACC_PUBLIC | ClassFile.ACC_STATIC,
                  codeBuilder ->
                      generateCode(ast, codeGeneratorEnvironment, codeBuilder).return_());
            });
  }
\end{lstlisting}
\caption{Entry point for the JVM code generator backend}
\label{fig:codegen_main}
\end{figure}

Depending on the nodes in the AST, we then end up in the appropriate case. For a simple case, let us consider the case
for value-expressions for the basic built-in types of Hygge, \texttt{UnitVal}, \texttt{BoolVal}, etc. For the \texttt{UnitVal}-case,
the simplest thing is to do nothing and return an unmodified code builder. The rest of the value expressions all follow a similar approach,
loading a constant onto the stack with the \texttt{ldc}\cite{jvm_spec} instruction, except for the \texttt{BoolVal} expression which makes use of the
\texttt{iconst} instructions instead. With the \texttt{ldc()}-method, the Class-file API places the value in the run-time constant pool
of the \texttt{HyggeMain} class and automatically inserts the correct reference to the constant in the pool. This is one of the ways
that the Class-file API provides abstraction while still giving the programmer control over exactly which instructions are generated.
In figure \ref{fig:jvm_value_codegen}, we show the simple implementation of JVM bytecode generation for Hygge values.

\begin{figure}[H]
\centering
\begin{lstlisting}[language=Java]
  // From JVMCodeGenerator.java
  public CodeBuilder generateCode(
      Node<E, T> ast, JVMCodeGeneratorEnvironment env, CodeBuilder codeBuilder) {
    return switch (ast.expr()) {
      case UnitVal<E, T> _ -> codeBuilder;
      case BoolVal<E, T> v -> v.value() ? codeBuilder.iconst_1() : codeBuilder.iconst_0();
      case IntVal<E, T> v -> codeBuilder.ldc(v.value());
      case FloatVal<E, T> v -> codeBuilder.ldc(v.value());
      case StringVal<E, T> v -> codeBuilder.ldc(v.value());
      // Remaining cases ... 
    }
  }
 
\end{lstlisting}
\caption{An excerpt from \texttt{JVMCodeGenerator.java} for the simple case of generating JVM bytecode for Hygge values using the Class-file API.}
\label{fig:jvm_value_codegen}
\end{figure}

This is rather simple, but it doesn't show the full potential of the Class-file API. A more intriguing example would be the code generation
for \texttt{if}-expressions for which the code is shown in figure \ref{fig:jvm_if_expressions}. To implement Hygge \texttt{if}-expressions on the JVM, labels are used. Similar to assembly, JVM bytecode
has branching instructions such as \texttt{ifne} and \texttt{goto}\cite{jvm_spec}. However, these branching instructions don't jump to a label, but rather an offset in bytes.
This offset is calculated from the number bytes taken up by every instruction and operand since the beginning of a method body. Calculating
these offsets manually would incur some bookkeeping on the programmer, but luckily the Class-file API provides the \texttt{Label}\cite{java_label} abstraction,
which accounts for the offsets automatically relieving this burden. Still, the Class-file API offers control as the methods for generating branching
instructions accept a \texttt{Label} as an argument and the \texttt{labelBinding()}-method\cite{java_codebuilder} allows the programmer to arbitrarily set the target for a label.
This makes for a straight-forward, concise and declarative implemention of Hygge \texttt{if}-expressions.

\begin{figure}[H]
\centering 
\begin{lstlisting}[language=Java]
// From JVMControlFlowCodeGenerator.java
public class JVMControlFlowCodeGenerator<E extends TypingEnvironment, T extends Type> {

  public CodeBuilder generateCode(
      JVMCodeGenerator<E, T> codeGenerator,
      JVMCodeGeneratorEnvironment env,
      CodeBuilder codeBuilder,
      Node<E, T> node,
      If<E, T> ifExpr) {
    Label labelTrue = codeBuilder.newLabel();
    Label labelFalse = codeBuilder.newLabel();

    // Generate code for the condition expression and jump to true branch
    // if condition is not 0
    codeBuilder = codeGenerator.generateCode(ifExpr.condition(), env, codeBuilder).ifne(labelTrue);

    // Generate code for false branch
    codeBuilder =
        codeGenerator
            .generateCode(ifExpr.ifFalse(), env, codeBuilder)
            .goto_(labelFalse)
            .labelBinding(labelTrue);

    // Generate code for true branch
    return codeGenerator.generateCode(ifExpr.ifTrue(), env, codeBuilder).labelBinding(labelFalse);
  }
 
  // Remaining methods ...
}
\end{lstlisting}
\caption{An excerpt from \texttt{JVMControlFlowCodeGenerator.java} showing code generation for Hygge \texttt{if}-expressions}
\label{fig:jvm_if_expressions}
\end{figure}

Generating JVM bytecode for functions becomes a bit trickier. The issue being that one cannot really create methods inside a method in the JVM,
at least not at the bytecode level. Here, what would be a lambda function in \texttt{Java} becomes a private static method on the parent class
of the current scope. Holding a reference to a function is also not as easy as one must use a \texttt{java.util.function.Function}\cite{java_function} instance and
this generic class cannot have primitive types as its type parameters, so this requires some boxing and unboxing with wrapper-classes such as \texttt{Integer}.
Additionally, when one wants to apply the function to some arguments one must push the reference to the \texttt{Function} instance to the top of
the stack and then invoke the \texttt{apply()}-method on it.

This is a lot of hoops to jump through. Rather than using all sorts of abstractions like \texttt{Function}, it is simpler to just generate static
methods and attach them to the parent class and register the function name and type signature in the \texttt{JVMCodeGeneratorEnvironment}, which
keeps track of information for JVM bytecode generation. This \texttt{JVMCodeGeneratorEnvironment} should also provide a way to reference the
parent class. Recall the earlier note on the difference in mutability between a \texttt{ClassBuilder} and \texttt{CodeBuilder} and it becomes
sensible to let the \texttt{JVMCodeGeneratorEnvironment} keep a reference to the builder for the parent class. When encountering
an application of the function/method, one can simply use the \texttt{invokestatic}\cite{jvm_spec} instruction without first having to push a reference
to some class to the top of the stack. In figure \ref{fig:jvm_function_builder}, the \texttt{generateCode()}-method is shown.

\begin{figure}[H]
 \centering 
 \begin{lstlisting}[language=Java]
  public CodeBuilder generateCode(
      JVMCodeGenerator<E, T> codeGenerator,
      JVMCodeGeneratorEnvironment env,
      CodeBuilder codeBuilder,
      Node<E, T> node,
      Lambda<E, T> lambdaExpr) {
    var methodEnv = env.with(new HashMap<>());
    for (var arg : lambdaExpr.args()) {
      methodEnv.store(
          arg.name(),
          pretypeResolver.resolvePretype(methodEnv.getTypingEnvironment(), arg.preType()));
    }
    env.getParentClass()
        .withMethodBody(
            env.getNameStack().peek(),
            ofDescriptor(
                JVMCodeGenerationUtility.makeDescriptor(
                    lambdaExpr.args().stream()
                        .map(
                            arg ->
                                pretypeResolver.resolvePretype(
                                    env.getTypingEnvironment(), arg.preType()))
                        .toList(),
                    lambdaExpr.body().type(),
                    env)),
            ClassFile.ACC_PUBLIC | ClassFile.ACC_STATIC,
            methodCodeBuilder ->
                JVMCodeGenerationUtility.inferReturn(
                    codeGenerator.generateCode(lambdaExpr.body(), methodEnv, methodCodeBuilder),
                    lambdaExpr.body().type(),
                    env));
    return codeBuilder;
  }
 \end{lstlisting}
 \caption{Excerpt of the implementation of the \texttt{generateCode()}-method handling Hygge \texttt{Lambda}-expressions in \texttt{JVMFunctionCodeGenerator.java}.}
 \label{fig:jvm_function_builder}
\end{figure}

Aside from that, all that is needed is to format some method and class descriptors in the form of strings. This is done using the pretype resolution strategy
implemented as part of the typechecking phase to infer the argument and return types of the function. Generating the JVM bytecode for the function
body is also easy as the code generator can be invoked on the lambdas body expression. As a final step, an appropriate return instruction is inferred
from the resolved function return type and is appended.

For the final high-light, we showcase the implementation of code generation for Hygge \texttt{struct}s in figure \ref{fig:jvm_struct_builder}.
This is the most problematic of all the Hygge expressions because of the structural type system of Hygge. In the \texttt{Java} language, and consequently the JVM,
type equivalence is determined nominally, i.e. on the name of two types. A type with some name can only be declared once and is thus unique, enabling
a simple check for equivalence. In a structural type system, equivalence is determined by the contents of a type; if two structs have the same fields with
the same types, regardless of the name of the structs, they are considered equivalent and thus interchangable. So how can this be reconciled?

The solution is simple: everything in \texttt{Java} is an \texttt{java.lang.Object}\cite{java_object} (except for primitives, of course). A \texttt{struct} is really
nothing more than a mapping between names and values, and such an abstraction already exists: the \texttt{java.util.HashMap}\cite{hashmap}.
Exploiting the fact that all classes in \texttt{Java} implicitly inherits from \texttt{java.lang.Object} means that we can
model structs as \texttt{HashMap<String, Object>} instances. To make this work with primitives whose types cannot be used for generics,
the solution is to ``box'' and ``unbox'' these values with wrapper classes like \texttt{java.lang.Integer} and \texttt{java.lang.Boolean}.
In figure \ref{fig:jvm_struct_builder}, we show the implementation of the JVM bytecode generator for Hygge \texttt{struct}s. This utilizes the \texttt{Map.of()}
method and the \texttt{HashMap} copy constructor for initialization along with the mentioned wrapper classes to ``box''
any primitives provided by the \texttt{JVMCodeGenerationUtility.boxPrimitive()} utility method to effectively implement
struct construction.

\begin{figure}[H]
  \centering
  \begin{lstlisting}[language=Java]
  public CodeBuilder generateCode(
      JVMCodeGenerator<E, T> codeGenerator,
      JVMCodeGeneratorEnvironment env,
      CodeBuilder codeBuilder,
      Node<E, T> node,
      StructCons<E, T> structConsExpr) {
    codeBuilder = codeBuilder.new_(ClassDesc.ofInternalName("java/util/HashMap")).dup();

    var numberOfFields = structConsExpr.fields().size();
    for (var field : structConsExpr.fields().entrySet()) {
      codeBuilder = codeBuilder.ldc(field.getKey());
      codeBuilder = codeGenerator.generateCode(field.getValue(), env, codeBuilder);
      var fieldType = TypeUtility.expandType(field.getValue().env(), field.getValue().type());
      codeBuilder = JVMCodeGenerationUtility.boxPrimitive(codeBuilder, fieldType);
    }

    return codeBuilder
        .invokestatic(
            ClassDesc.ofInternalName("java/util/Map"),
            "of",
            MethodTypeDesc.ofDescriptor(
                "("
                    + "Ljava/lang/Object;Ljava/lang/Object;".repeat(numberOfFields)
                    + ")Ljava/util/Map;"),
            true)
        .invokespecial(
            ClassDesc.ofInternalName("java/util/HashMap"),
            "<init>",
            MethodTypeDesc.ofDescriptor("(Ljava/util/Map;)V"));
  }
  \end{lstlisting}
  \caption{Excerpt of the implementation of code generation for Hygge \texttt{struct}s from \texttt{JVMStructCodeGenerator.java}.}
  \label{fig:jvm_struct_builder}
\end{figure}

This makes the subsequent implementation of Hygge field selections straight-forward as the \texttt{HashMap}-API
provides the convenient \texttt{get} method for retrieving values and the \texttt{Object} instance can accurately be
casted using the typing information ascribed to each \texttt{Node} during the typechecking phase.
In figure \ref{fig:jvm_constructor_builder}, we show the implementation of code generation for Hygge field selections.

\begin{figure}
    \centering
    \begin{lstlisting}[language=Java]
  public CodeBuilder generateCode(
      JVMCodeGenerator<E, T> codeGenerator,
      JVMCodeGeneratorEnvironment env,
      CodeBuilder codeBuilder,
      Node<E, T> node,
      FieldSelect<E, T> fieldSelectExpr) {
    codeBuilder = codeGenerator.generateCode(fieldSelectExpr.target(), env, codeBuilder);
    codeBuilder =
        codeBuilder
            .ldc(fieldSelectExpr.field())
            .invokevirtual(
                ClassDesc.ofInternalName("java/util/HashMap"),
                "get",
                MethodTypeDesc.ofDescriptor("(Ljava/lang/Object;)Ljava/lang/Object;"));

    var fieldType = TypeUtility.expandType(node.env(), node.type());

    return JVMCodeGenerationUtility.checkAndUnbox(codeBuilder, fieldType);
  }
    \end{lstlisting}
    \caption{Implementation of the code generator for field selections from \texttt{JVMStructCodeGenerator.java}.}
    \label{fig:jvm_constructor_builder}
\end{figure}

Similarly, the case for assigning values to a field in a struct is solved by using the \texttt{put()}-method provided
by the \texttt{HashMap}-API\cite{hashmap}. One thing to note is that the semantics of the Hygge language require that the right-hand side
of an assignment is the result of such an expression. Doing so is luckily also simple as the code generation for field selection
retrieves the value using the \texttt{get()}-method. One thing to be aware of it that the \texttt{put()}-method returns
the old value corresponding to a key, or \texttt{null} in the case that the key wasn't in the map before. This value
is of no use, so it must be popped of the stack using the \texttt{pop}-instruction\cite{jvm_spec}. An excerpt of the implemention
of code generation for assignment expressions can be seen on figure \ref{fig:jvm_field_assignment}.

\begin{figure}[H]
  \centering
  \begin{lstlisting}[language=Java]
  // From JVMCodeGenerator.java
  private CodeBuilder generateCode(
      Node<E, T> ast,
      JVMCodeGeneratorEnvironment env,
      CodeBuilder codeBuilder,
      Assign<E, T> assignExpr) {
    return switch (assignExpr.target().expr()) {
      case Var<E, T> v -> // Redacted 
      
      case FieldSelect<E, T> v -> {
        codeBuilder = generateCode(v.target(), env, codeBuilder).ldc(v.field());
        codeBuilder = generateCode(assignExpr.rhs(), env, codeBuilder);

        var rhsType = TypeUtility.expandType(assignExpr.rhs().env(), assignExpr.rhs().type());

        codeBuilder = JVMCodeGenerationUtility.boxPrimitive(codeBuilder, rhsType);

        codeBuilder =
            codeBuilder
                .invokevirtual(
                    ClassDesc.ofInternalName("java/util/HashMap"),
                    "put",
                    MethodTypeDesc.ofDescriptor(
                        "(Ljava/lang/Object;Ljava/lang/Object;)Ljava/lang/Object;"))
                .pop();

        yield structCodeGenerator.generateCode(this, env, codeBuilder, ast, v);
      }
      default -> // Redacted
    };
  }
   
  \end{lstlisting}
  \caption{Excerpt of the implemention of code generation for assignment expressions with an emphasis on field selections.}
  \label{fig:jvm_field_assignment}
\end{figure}
