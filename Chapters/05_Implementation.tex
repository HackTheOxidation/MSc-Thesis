\chapter{Implementation}

At this point, we have arrived at a suitable architecture and design for \texttt{JHygge}, the new Hygge compiler, so this chapter concerns the actual
implementation. Here, we emphasize the use of the new functional features of modern Java (JDK 21 and up) and present the highlights
of the final implementation of \texttt{JHygge}. 

\section{Use of Functional Java: Records, Sealed Interfaces, Pattern matching and Monads}

As the purpose of this thesis is also to explore the newer functional features of \texttt{Java}. The existing implementation of \texttt{hyggec}
is indeed very functional, so we attempted to write the implementations of both the AST, typechecker and interpreter in such a way that
the \texttt{JHygge} versions are as close to those of \texttt{hyggec}, disregarding the architectural and design differences. Some of
the newer functional features of \texttt{Java} we wanted to investigate include records (immutable classes), sealed interfaces (discriminated unions),
switch expressions (pattern matching) and the introduction of some monads, most notably \texttt{Optional}. In the following sections,
we show concrete examples of implementation, where we have attempted to use said features and compare them to the corresponding code
in \texttt{hyggec}.

\subsection{Records and Sealed Interfaces for functional data modelling}

The \texttt{Node} and \texttt{Expr} data model for the AST from \texttt{hyggec} is one of the parts that we wish to carry over in \texttt{JHygge}.
This requires support for discriminated unions and recursive data type, the former of which was introduced in JDK 17 in the form of sealed
interfaces; one declares a finite number of exact specializations upon defining the interface itself. This gives the \texttt{Java} compiler
information about the possible specialization, which it can use to perform case analysis and thus enforce exhaustive pattern matching. Let's
see records and sealed interfaces in action in the form of \texttt{JHygge}s implementation of \texttt{Expr} in the listing below:

\begin{lstlisting}[language=Java]
// From Expr.java, reformatted to be more compact and fit within the margins
public sealed interface Expr<E, T>
    permits Expr.UnitVal, Expr.BoolVal, Expr.IntVal, Expr.FloatVal,
        Expr.StringVal, Expr.Var, Expr.BinaryOperator, Expr.Not,
        Expr.ReadInt, Expr.ReadFloat, Expr.Print, Expr.PrintLn,
        Expr.If, Expr.Seq, Expr.Type, Expr.Ascription,
        Expr.Assertion, Expr.Let, Expr.Assign, Expr.While,
        Expr.Lambda, Expr.Application, Expr.StructCons,
        Expr.FieldSelect, Expr.Pointer, Expr.UnionCons,
        Expr.Match {
    // ...
}
\end{lstlisting}

Similar to discriminated unions in \texttt{F\#}, all cases are declared for the sealed interface \texttt{Expr}. However, unlike \texttt{F\#},
the cases have to be defined as separate classes, records or interfaces. In the case that a specialization is a class, they have to be \texttt{final}.

\begin{lstlisting}[language=Java]
  // From Expr.java, Not<E, T> is an inner class of Expr<E, T>
  record Not<E, T>(Node<E, T> arg) implements Expr<E, T> {
    @Override
    public String prettyPrint(int indentation) {
      indentation += 2;
      return "Not(arg=" + arg.expr().prettyPrint(indentation) + ")";
    }
  }
\end{lstlisting}

\subsection{Switch expressions for case-by-case definitions and dispatching}

\begin{lstlisting}[language=Java]
  // From StandardHyggeInterpreter.java, slightly reformatted
  @Override
  public Optional<InterpreterResult<E, T>> reduce(RuntimeEnvironment<E, T> env, Node<E, T> node) {
    return switch (node.expr()) {
      case UnitVal<E, T> _,
          BoolVal<E, T> _,
          IntVal<E, T> _,
          FloatVal<E, T> _,
          StringVal<E, T> _,
          Pointer<E, T> _ ->
          Optional.empty();
      case Var<E, T> v when env.isMutable(v.name()) ->
          InterpreterResult.of(env, env.mutables().get(v.name()));
      case Var<E, T> _ -> Optional.empty();
      case BinaryOperator<E, T> v -> binaryOperatorReducer.reduce(this, env, node, v);
      case Not<E, T> v ->
          (v.arg().expr() instanceof BoolVal<E, T> b)
              ? InterpreterResult.of(env, node.with(b.not()))
              : reduce(env, v.arg())
                  .map(a -> new InterpreterResult<>(env, node.with(new Not<>(a.node()))));
      case ReadInt<E, T> v -> ioReducer.reduce(this, env, node, v);
      case ReadFloat<E, T> v -> ioReducer.reduce(this, env, node, v);
      case Print<E, T> v -> ioReducer.reduce(this, env, node, v);
      case PrintLn<E, T> v -> ioReducer.reduce(this, env, node, v);
      case If<E, T> v -> controlFlowReducer.reduce(this, env, node, v);
      case Seq<E, T> v -> controlFlowReducer.reduce(this, env, node, v);
      case Type<E, T> v -> letReducer.reduce(this, env, node, v);
      case Ascription<E, T> v -> letReducer.reduce(this, env, node, v);
      case Assertion<E, T> v -> controlFlowReducer.reduce(this, env, node, v);
      case Let<E, T> v -> letReducer.reduce(this, env, node, v);
      case Assign<E, T> v -> letReducer.reduce(this, env, node, v);
      case While<E, T> v -> controlFlowReducer.reduce(this, env, node, v);
      case Application<E, T> v -> functionReducer.reduce(this, env, node, v);
      case Lambda<E, T> v -> functionReducer.reduce(this, env, node, v);
      case StructCons<E, T> v -> structReducer.reduce(this, env, node, v);
      case FieldSelect<E, T> v -> structReducer.reduce(this, env, node, v);
      case UnionCons<E, T> v -> unionReducer.reduce(this, env, node, v);
      case Match<E, T> v -> unionReducer.reduce(this, env, node, v);
    };
  }
\end{lstlisting}

\begin{lstlisting}[language=Java]
  // From StandardHyggeControlFlowReducer.java
  @Override
  public Optional<InterpreterResult<E, T>> reduce(
      HyggeInterpreter<E, T> interpreter,
      RuntimeEnvironment<E, T> env,
      Node<E, T> node,
      Seq<E, T> seqExpr) {
    var nodes = seqExpr.nodes();

    if (nodes.isEmpty()) {
      return InterpreterResult.of(env, node.with(new UnitVal<>()));
    }

    if (nodes.size() == 1) {
      var last = nodes.getFirst();
      return ASTUtility.isValue(last)
          ? InterpreterResult.of(env, node.with(last.expr()))
          : interpreter
              .reduce(env, last)
              .flatMap(
                  result -> InterpreterResult.of(result.env(), node.with(result.node().expr())));
    }

    var first = nodes.getFirst();
    var rest = nodes.subList(1, nodes.size());
    return !ASTUtility.isValue(first)
        ? interpreter
            .reduce(env, first)
            .flatMap(
                result ->
                    InterpreterResult.of(
                        result.env(),
                        node.with(
                            new Seq<>(
                                Stream.concat(Stream.of(result.node()), rest.stream()).toList()))))
        : InterpreterResult.of(env, node.with(new Seq<>(rest)));
  }
\end{lstlisting}

\subsection{\texttt{Optional} for monadic error handling}

\section{Parsing Hygge source files with ANTLR4}

As mentioned in the chapter on Architecture and Design, we have selected ANTLR4 as the parser generator of choice. For starters,
we have translated the \texttt{FsLexYacc} lexer and parser files, \texttt{Lexer.fsl} and \texttt{Parser.fsy}, to ANTLR4
\texttt{.g4}-format. In the listing below, we show the complete \texttt{.g4} grammar file for the final version of \texttt{JHygge}.

\begin{lstlisting}
grammar Hygge;

@header {
package io.github.hacktheoxidation.hygge;
}

prog: expression EOF;

expression: TYPE ID EQUAL pretype SEQ expression                                        # TypeAlias
    | LET (MUTABLE)? variable (ASCRIPTION pretype)? EQUAL simpleExpr SEQ expression     # LetDeclare
    | FUN variable parenArgTypesSeq ASCRIPTION pretype EQUAL simpleExpr SEQ expression  # FunDeclare
    | curlyExpr expression                                                              # Curly
    | sequenceExpr                                                                      # Seq
    ;

curlyExpr: LCURLY expression RCURLY                                       # Scope
    ;

sequenceExpr: simpleExpr SEQ expression                                   # Sequence
    | simpleExpr                                                          # Simple
    ;

simpleExpr: ifExpr                                                        # If
    | curlyExpr                                                           # Brackets
    | WHILE simpleExpr DO simpleExpr                                      # While
    | primaryExpr LARROW simpleExpr                                       # Assignment
    | FUN parenArgTypesSeq RARROW simpleExpr                              # Lambda
    | MATCH simpleExpr WITH LCURLY matchCases RCURLY                      # Match
    ;

ifExpr: IF simpleExpr THEN simpleExpr ELSE simpleExpr                     # IfThenElse
    | orExpr                                                              # Or
    ;

orExpr: orExpr OR andExpr                                                 # LogicalOr
    | andExpr                                                             # And
    ;

andExpr: andExpr AND relExpr                                              # LogicalAnd
    | relExpr                                                             # Rel
    ;

relExpr: addExpr EQUAL addExpr                                            # Equality
    | addExpr LESS addExpr                                                # RelLess
    | addExpr                                                             # Add
    ;

addExpr: addExpr ADD multExpr                                             # ArithAdd
    | multExpr                                                            # Mult
    ;

multExpr: multExpr MUL unaryExpr                                          # ArithMul
    | unaryExpr                                                           # Unary
    ;

unaryExpr: NOT unaryExpr                                                  # LogicalNot
    | READINT ('(' ')' | '()')                                            # ReadInt
    | READFLOAT ('(' ')' | '()')                                          # ReadFloat
    | PRINT '(' simpleExpr ')'                                            # Print
    | PRINTLN '(' simpleExpr ')'                                          # PrintLine
    | ASSERT '(' simpleExpr ')'                                           # Assertion
    | ascriptionExpr                                                      # Ascription
    | primaryExpr parenExprCommaSeq                                       # Application
    ;

ascriptionExpr: primaryExpr ASCRIPTION pretype                            # TypeAscription
    | primaryExpr                                                         # Primary
    ;

primaryExpr: LPAREN simpleExpr RPAREN                                     # Parens
    | value                                                               # Val
    | variable                                                            # Var
    | STRUCT LCURLY structFieldInitSeq RCURLY                             # StructCons
    | primaryExpr DOT field                                               # FieldSelect
    | label LCURLY expression RCURLY                                      # UnionCons
    ;


value : INT | FLOAT | BOOL | STRING | UNIT ;
variable : ID ;
field : ID ;
label: ID ;

pretype : parenTypesSeq RARROW pretype
    | STRUCT LCURLY structFieldTypeSeq RCURLY
    | UNION LCURLY unionLabelTypeSeq RCURLY
    | ID
    ;


parenArgTypesSeq: UNIT
    | LPAREN (variable ASCRIPTION pretype (COMMA variable ASCRIPTION pretype)*) RPAREN
    ;

parenTypesSeq: UNIT
    | LPAREN (pretype (COMMA pretype)*) RPAREN
    ;

parenExprCommaSeq: UNIT
    | LPAREN (simpleExpr (COMMA simpleExpr)*) RPAREN
    ;

structFieldInitSeq: field EQUAL simpleExpr (SEQ field EQUAL simpleExpr)*;

structFieldTypeSeq: field ASCRIPTION pretype (SEQ field ASCRIPTION pretype)*;

unionLabelTypeSeq: label ASCRIPTION pretype (SEQ label ASCRIPTION pretype)*;

matchCases: label LCURLY variable RCURLY RARROW simpleExpr (SEQ label LCURLY variable RCURLY RARROW simpleExpr)*;

// Reserved keywords and operators
ASSERT : 'assert' ;
READINT : 'readInt' ;
READFLOAT : 'readFloat' ;
PRINT : 'print' ;
PRINTLN : 'println';
NOT : 'not' ;
ADD : '+' ;
SUB : '-' ;
MUL : '*' ;
LESS : '<' ;
AND : 'and' ;
OR : 'or' ;
IF : 'if' ;
THEN : 'then' ;
ELSE : 'else' ;
LET : 'let' ;
TYPE : 'type' ;
EQUAL : '=' ;
LPAREN : '(' ;
RPAREN : ')' ;
LCURLY : '{' ;
RCURLY : '}' ;
ASCRIPTION : ':' ;
SEQ : ';' ;
COMMA : ',' ;
DOT : '.' ;
LARROW : '<-' ;
RARROW : '->' ;
MUTABLE : 'mutable' ;
WHILE : 'while' ;
DO : 'do' ;
FUN : 'fun' ;
STRUCT : 'struct' ;
UNION : 'union' ;
MATCH : 'match' ;
WITH : 'with' ;

// Values and Names
INT : [0-9]+ ;
FLOAT : INT '.' INT ('f' | 'F') ;
BOOL : 'true' | 'false' ;
STRING : '"' (~["\\\r\n] | EscapeSequence)* '"' ;
UNIT : '()' ;
ID: [a-zA-Z_][a-zA-Z_0-9]* ;
WS: [ \t\n\r\f]+ -> skip ;
LINE_COMMENT: '//' ~[\r\n]* -> skip ;

fragment EscapeSequence: '\\' [btnfr"'\\] ; 
\end{lstlisting}

Initially, this grammar file was syntactically more similar to the E-BNF for the \texttt{Hygge} specification as an attempt to
simplify the grammar file as much as possible. However, the \texttt{Hygge} grammar specification does not take operator
precedence into account, so it was later decided to factor out left-recursion and enforce operator precedence by separating
production rules similar to that of \texttt{Parser.fsy} in \texttt{hyggec}.

By default, \texttt{ANTLR4} generates both a lexer and a parser along with classes for implementing listeners and visitors.
One could implement the rest of \texttt{JHygge} using the \texttt{ANTLR4} AST representation, but this AST is intended to
be consumed or transformed by listeners and visitors, respecitively. Instead, we want a more functional model of an AST.
In order to breach the gap between the \texttt{ANTLR4} representation and our desired \texttt{Node} and \texttt{Expr} data model,
we decided to create a visitor, \texttt{AntlrHyggeNodeVisitor}, that provides this transformation. In the listing below, we
see an excerpt of \texttt{AntlrHyggeNodeVisitor}, namely the \texttt{visit}, which transforms a \texttt{HyggeParser.AssertionContext}
to a \texttt{Node} with an \texttt{Expr.Assertion}-expression, as well as the code for the \texttt{Expr.Assertion} record.

\begin{lstlisting}[language=Java]
 
  // From AntlrHyggeNodeVisitor.java
  @Override
  public Node<Object, Object> visitAssertion(HyggeParser.AssertionContext ctx) {
    return new Node<>(
        new Expr.Assertion<>(ctx.simpleExpr().accept(this)),
        Position.from(ctx.start, ctx.stop),
        null,
        null);
  }

  // From Expr.java
  record Ascription<E, T>(PreTypeNode tpe, Node<E, T> node) implements Expr<E, T> {
    @Override
    public String prettyPrint(int indentation) {
      return "Ascription(node="
          + node.expr().prettyPrint(indentation + 2)
          + ", tpe="
          + tpe.preType().toString()
          + ")";
    }
  }

\end{lstlisting}

The creation of \texttt{Node}- and \texttt{Expr}-objects is mostly straight forward even though they are mutually recursive.
To parse the argument for the assertion, one simply calls the \texttt{accept()}-method on the subexpression for the visitor
context, which will recursively traverse and transform the AST to \texttt{Node}-instances. Most of the expression types
follow a similar implementation style as the one for assertions. There are some of these tranformations that are more complicated,
like function declarations, structs, unions, etc., but it is the same case for declaring the equivalent parser for \texttt{hyggec}
using \texttt{FsLexYacc}.

To compare the two parser implementations in a more quantitative way, the parser in \texttt{JHygge}
takes up 159 lines (\texttt{hygge.g4}) and 440 lines (\texttt{AntlrHyggeNodeVisitor}) for a total of 599 lines, where the lexer and
parser for \texttt{hyggec} is 100 lines (\texttt{Lexer.fsl}) and 305 lines (\texttt{Parser.fsy}) totalling 405 lines. The Java
implementation takes up approx. $48\%$ more lines than the \texttt{F\#} version, where the \texttt{AntlrHyggeNodeVisitor} is
the largest contributor.

\section{Generating JVM bytecode with the Class-file API from Java 24}

The Class-file API promises a declarative approach to reading, modifying and writing \texttt{.class}-files. The main parts of the
API used in the implementation of the JVM backend in \texttt{JHygge} is the \texttt{ClassBuilder} and the \texttt{CodeBuilder}.
A \texttt{ClassBuilder} is intended for modifying classes at a higher-level, such as adding attributes, constructors and methods,
without directly handling bytecode. A \texttt{CodeBuilder} is then used to construct the sequence of bytecode instructions in
a method or a constructor. Both the APIs for \texttt{ClassBuilder}s and \texttt{CodeBuilder}s make use of the builder pattern,
however they are different in the respect that a \texttt{CodeBuilder} is immutable whereas a \texttt{ClassBuilder} isn't.
We utilize this difference in mutability during the code generation of functions and structs.

Let's walk through an example to show, how the Class-file API is used concretely in \texttt{JHygge}. The code listing below shows
the implementation of the \texttt{generateCode()}-method for the \texttt{JVMCodeGenerator}-class. This is the minimal requirement
for implementing the \texttt{HyggeCodeGenerator}-interface. As we see in the listing, we initially generate an empty \texttt{.class}-file,
which is always \texttt{HyggeMain.class}, containing the \texttt{HyggeMain} class. The \texttt{ClassBuilder}-API is then used to
generate the static \texttt{main()}-method, which the JVM will look for and use as an entry-point. To generate the code for
the body of the \texttt{main()}-method, we use the immutable \texttt{CodeBuilder}-API. Similar to the typechecker and the interpreter,
code generation is split across different classes and methods by grouping language features together, so the overloaded and private
\texttt{generateCode()}-method will then dispatch to the correct code generator method depending on the top-level node in the AST.
Note also, that the \texttt{JVMCodeGeneratorEnvironment}-class keeps a reference to the parent \texttt{ClassBuilder} for the
\texttt{HyggeMain}-class. This will come in handy later, when generating code for functions, for example.

\begin{lstlisting}[language=Java]
  // From JVMCodeGenerator.java
  @Override
  public void generateCode(Node<E, T> ast, String fileName) throws IOException {
    var filePath = Path.of("HyggeMain.class").toAbsolutePath();
    System.out.println("Compiling '" + fileName + "' to: " + filePath);
    ClassFile.of()
        .buildTo(
            filePath,
            ClassDesc.of("HyggeMain"),
            classBuilder -> {
              var codeGeneratorEnvironment = new JVMCodeGeneratorEnvironment(classBuilder);
              classBuilder.withMethodBody(
                  "main",
                  ofDescriptor("([Ljava/lang/String;)V"),
                  ClassFile.ACC_PUBLIC | ClassFile.ACC_STATIC,
                  codeBuilder ->
                      generateCode(ast, codeGeneratorEnvironment, codeBuilder).return_());
            });
  }
\end{lstlisting}

Depending on the nodes in the AST, we then end up in the appropriate case. For a simple case, 
