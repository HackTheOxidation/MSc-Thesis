\chapter{Appendix A - ANTLR4 grammar for Hygge} \label{appendix:hygge_grammar}

\begin{lstlisting}
grammar Hygge;

@header {
package io.github.hacktheoxidation.hygge;
}

prog: expression EOF;

expression: TYPE ID EQUAL pretype SEQ expression                                        # TypeAlias
    | LET (MUTABLE)? variable (ASCRIPTION pretype)? EQUAL simpleExpr SEQ expression     # LetDeclare
    | FUN variable parenArgTypesSeq ASCRIPTION pretype EQUAL simpleExpr SEQ expression  # FunDeclare
    | curlyExpr expression                                                              # Curly
    | sequenceExpr                                                                      # Seq
    ;

curlyExpr: LCURLY expression RCURLY                                       # Scope
    ;

sequenceExpr: simpleExpr SEQ expression                                   # Sequence
    | simpleExpr                                                          # Simple
    ;

simpleExpr: ifExpr                                                        # If
    | curlyExpr                                                           # Brackets
    | WHILE simpleExpr DO simpleExpr                                      # While
    | primaryExpr LARROW simpleExpr                                       # Assignment
    | FUN parenArgTypesSeq RARROW simpleExpr                              # Lambda
    | MATCH simpleExpr WITH LCURLY matchCases RCURLY                      # Match
    ;

ifExpr: IF simpleExpr THEN simpleExpr ELSE simpleExpr                     # IfThenElse
    | orExpr                                                              # Or
    ;

orExpr: orExpr OR andExpr                                                 # LogicalOr
    | andExpr                                                             # And
    ;

andExpr: andExpr AND relExpr                                              # LogicalAnd
    | relExpr                                                             # Rel
    ;

relExpr: addExpr EQUAL addExpr                                            # Equality
    | addExpr LESS addExpr                                                # RelLess
    | addExpr                                                             # Add
    ;

addExpr: addExpr ADD multExpr                                             # ArithAdd
    | multExpr                                                            # Mult
    ;

multExpr: multExpr MUL unaryExpr                                          # ArithMul
    | unaryExpr                                                           # Unary
    ;

unaryExpr: NOT unaryExpr                                                  # LogicalNot
    | READINT ('(' ')' | '()')                                            # ReadInt
    | READFLOAT ('(' ')' | '()')                                          # ReadFloat
    | PRINT '(' simpleExpr ')'                                            # Print
    | PRINTLN '(' simpleExpr ')'                                          # PrintLine
    | ASSERT '(' simpleExpr ')'                                           # Assertion
    | ascriptionExpr                                                      # Ascription
    | primaryExpr parenExprCommaSeq                                       # Application
    ;

ascriptionExpr: primaryExpr ASCRIPTION pretype                            # TypeAscription
    | primaryExpr                                                         # Primary
    ;

primaryExpr: LPAREN simpleExpr RPAREN                                     # Parens
    | value                                                               # Val
    | variable                                                            # Var
    | STRUCT LCURLY structFieldInitSeq RCURLY                             # StructCons
    | primaryExpr DOT field                                               # FieldSelect
    | label LCURLY expression RCURLY                                      # UnionCons
    ;


value : INT | FLOAT | BOOL | STRING | UNIT ;
variable : ID ;
field : ID ;
label: ID ;

pretype : parenTypesSeq RARROW pretype
    | STRUCT LCURLY structFieldTypeSeq RCURLY
    | UNION LCURLY unionLabelTypeSeq RCURLY
    | ID
    ;


parenArgTypesSeq: UNIT
    | LPAREN (variable ASCRIPTION pretype (COMMA variable ASCRIPTION pretype)*) RPAREN
    ;

parenTypesSeq: UNIT
    | LPAREN (pretype (COMMA pretype)*) RPAREN
    ;

parenExprCommaSeq: UNIT
    | LPAREN (simpleExpr (COMMA simpleExpr)*) RPAREN
    ;

structFieldInitSeq: field EQUAL simpleExpr (SEQ field EQUAL simpleExpr)*;

structFieldTypeSeq: field ASCRIPTION pretype (SEQ field ASCRIPTION pretype)*;

unionLabelTypeSeq: label ASCRIPTION pretype (SEQ label ASCRIPTION pretype)*;

matchCases: label LCURLY variable RCURLY RARROW simpleExpr (SEQ label LCURLY variable RCURLY RARROW simpleExpr)*;

// Reserved keywords and operators
ASSERT : 'assert' ;
READINT : 'readInt' ;
READFLOAT : 'readFloat' ;
PRINT : 'print' ;
PRINTLN : 'println';
NOT : 'not' ;
ADD : '+' ;
SUB : '-' ;
MUL : '*' ;
LESS : '<' ;
AND : 'and' ;
OR : 'or' ;
IF : 'if' ;
THEN : 'then' ;
ELSE : 'else' ;
LET : 'let' ;
TYPE : 'type' ;
EQUAL : '=' ;
LPAREN : '(' ;
RPAREN : ')' ;
LCURLY : '{' ;
RCURLY : '}' ;
ASCRIPTION : ':' ;
SEQ : ';' ;
COMMA : ',' ;
DOT : '.' ;
LARROW : '<-' ;
RARROW : '->' ;
MUTABLE : 'mutable' ;
WHILE : 'while' ;
DO : 'do' ;
FUN : 'fun' ;
STRUCT : 'struct' ;
UNION : 'union' ;
MATCH : 'match' ;
WITH : 'with' ;

// Values and Names
INT : [0-9]+ ;
FLOAT : INT '.' INT ('f' | 'F') ;
BOOL : 'true' | 'false' ;
STRING : '"' (~["\\\r\n] | EscapeSequence)* '"' ;
UNIT : '()' ;
ID: [a-zA-Z_][a-zA-Z_0-9]* ;
WS: [ \t\n\r\f]+ -> skip ;
LINE_COMMENT: '//' ~[\r\n]* -> skip ;

fragment EscapeSequence: '\\' [btnfr"'\\] ; 
\end{lstlisting}

\chapter{Appendix B - Proposal for an implementation of the Result monad in Java 24} \label{appendix:result_java}

\begin{lstlisting}[language=Java]
package io.github.hacktheoxidation.util;

import java.util.function.Function;

/**
 * An implementation of the Result/Either monad in Java
 * @param <ValueType> The type of the Ok type constructor
 * @param <ErrorType> The type of the Error type constructor
 */
public sealed interface Result<ValueType, ErrorType> permits Result.Ok, Result.Error {

    /**
     * The functor map operation for the Result functor
     * @param function The mapping function which acts on the Ok type constructor
     * @return A new result instance
     * @param <NewValueType> The new polymorphic type for the Ok type constructor
     */
    <NewValueType> Result<NewValueType, ErrorType> map(Function<ValueType, NewValueType> function);

    /**
     * The monadic 'bind' (flatMap) operation for the Result monad
     * @param function A function from the Ok constructors value type to a new Result
     * @return A new result instance with the mapping applied
     * @param <NewValueType> The new polymorphic type for the Ok type constructor
     */
    <NewValueType> Result<NewValueType, ErrorType> flatMap(Function<ValueType, Result<NewValueType, ErrorType>> function);

    /**
     * The applicative functor 'liftA2' operation for the Result applicative
     * @param applicative A result applicative whose polymorphic type is a mapping function
     * @return A new result instance with the mapping applied
     * @param <NewValueType> The new polymorphic type for the Ok type constructor
     */
    <NewValueType> Result<NewValueType, ErrorType> lift(Result<Function<ValueType, NewValueType>, ErrorType> applicative);

    /**
     * Check if the result is an Ok instance
     * @return whether this is an Ok instance
     */
    default boolean isOk() {
        return this instanceof Result.Ok<?, ?>;
    }

    /**
     * Check if the result is an Error instance
     * @return wheterh this is an Error instance
     */
    default boolean isError() {
        return this instanceof Result.Error<?, ?>;
    }

    /**
     * Apply a procedure if this is an instance of the Ok constructor.
     * @param procedure A procedure/method that acts on the value of the Ok constructor
     */
    default void ifPresent(Function<ValueType, Void> procedure) {
        if (this instanceof Ok<ValueType, ErrorType>(ValueType value)) {
            procedure.apply(value);
        }
    }

    /**
     * The Ok type constructor for the Result monad signaling that a computation succeeded
     * @param value The value of the successful computation
     * @param <ValueType> The type of the value in the Ok constructor
     * @param <ErrorType> The type of the error in the Error constructor
     */
    record Ok<ValueType, ErrorType>(ValueType value) implements Result<ValueType, ErrorType> {
        @Override
        public <NewValueType> Result<NewValueType, ErrorType> map(Function<ValueType, NewValueType> function) {
            return new Ok<>(function.apply(value));
        }

        @Override
        public <NewValueType> Result<NewValueType, ErrorType> flatMap(Function<ValueType, Result<NewValueType, ErrorType>> function) {
            return function.apply(value);
        }

        @Override
        public <NewValueType> Result<NewValueType, ErrorType> lift(Result<Function<ValueType, NewValueType>, ErrorType> applicative) {
            return switch (applicative) {
                case Result.Ok<Function<ValueType, NewValueType>, ErrorType> ok -> new Ok<>(ok.value.apply(value()));
                case Result.Error<?, ErrorType> error -> new Error<>(error.error());
            };
        }
    }

    /**
     * The Error type constructor for the Result monad signaling that a computation failed
     * @param error The error of the failed computation
     * @param <ValueType> The type of the value in the Ok constructor
     * @param <ErrorType> The type of the error in the Error constructor
     */
    record Error<ValueType, ErrorType>(ErrorType error) implements Result<ValueType, ErrorType> {
        @Override
        public <NewValueType> Result<NewValueType, ErrorType> map(Function<ValueType, NewValueType> function) {
            return new Error<>(error);
        }

        @Override
        public <NewValueType> Result<NewValueType, ErrorType> flatMap(Function<ValueType, Result<NewValueType, ErrorType>> function) {
            return new Error<>(error);
        }

        @Override
        public <NewValueType> Result<NewValueType, ErrorType> lift(Result<Function<ValueType, NewValueType>, ErrorType> applicative) {
            return new Error<>(error);
        }
    }
}
    
\end{lstlisting}

\chapter{Appendix C - Additional functional utilities missing from Java 24} \label{appendix:functionalutilities}

\begin{lstlisting}[language=Java]
package io.github.hacktheoxidation.util;

import java.util.*;
import java.util.AbstractMap.SimpleEntry;
import java.util.function.BiFunction;
import java.util.stream.Collectors;
import java.util.stream.IntStream;
import java.util.stream.Stream;

/**
 * A collection of functions and utilities for Functional Programming (FP) in Java. Mostly for
 * manipulating collections in a more functional and immutable manor.
 */
public class FunctionalUtilities {
  private FunctionalUtilities() {}

  /**
   * Computes the union of two sets without modifying them.
   *
   * @param s1 The left-hand set operand
   * @param s2 The right-hand set operand
   * @return The union of 's1' and 's2'
   * @param <T> The type of elements in the sets
   */
  public static <T> Set<T> union(Set<T> s1, Set<T> s2) {
    return Stream.concat(s1.stream(), s2.stream()).collect(Collectors.toSet());
  }

  /**
   * Computes the union of multiple sets without modifying them.
   *
   * @param s1 The initial set operand
   * @param sets A variadic number of sets
   * @return The union of all set arguments
   * @param <T> The type of elements in the sets
   */
  @SafeVarargs
  public static <T> Set<T> union(Set<T> s1, Set<T>... sets) {
    return Arrays.stream(sets).reduce(s1, FunctionalUtilities::union);
  }

  /**
   * Computes the intersection of two sets without modifying them.
   *
   * @param s1 The left-hand set operand
   * @param s2 The right-hand set operand
   * @return The intersection of 's1' and 's2'
   * @param <T> The type of elements in the sets
   */
  public static <T> Set<T> intersection(Set<T> s1, Set<T> s2) {
    return s1.stream().filter(s2::contains).collect(Collectors.toSet());
  }

  /**
   * Computes the difference of two sets without modifying them.
   *
   * @param s1 The left-hand set operand
   * @param s2 The right-hand set operand
   * @return The difference of 's1' and 's2'
   * @param <T> The type of elements in the sets
   */
  public static <T> Set<T> difference(Set<T> s1, Set<T> s2) {
    return s1.stream().filter(val -> !s2.contains(val)).collect(Collectors.toSet());
  }

  /**
   * Tries to look up a key in a map
   *
   * @param m The map to search
   * @param key The key to look up
   * @return An optional with the corresponding value
   * @param <K> The type of the key in the map
   * @param <V> The type of the value in the map
   */
  public static <K, V> Optional<V> tryFind(Map<K, V> m, K key) {
    return m.containsKey(key) ? Optional.of(m.get(key)) : Optional.empty();
  }

  /**
   * Tries to look up a value in a set
   *
   * @param s The set to search
   * @param value The value to look up
   * @return An optional with the corresponding value
   * @param <V> The type of the value in the set
   */
  public static <V> Optional<V> tryFind(Set<V> s, V value) {
    return s.contains(value) ? Optional.of(value) : Optional.empty();
  }

  /**
   * Adds an element to a set
   *
   * @param s the set to add an element to
   * @param value the element to add
   * @return a new set with the element added
   * @param <V> the type of elements in the set
   */
  public static <V> Set<V> add(Set<V> s, V value) {
    var newSet = new HashSet<>(s);
    newSet.add(value);
    return newSet;
  }

  /**
   * Removes an element for a set
   *
   * @param s the set to remove an element from
   * @param value the element to remove
   * @return a new set with the element removed
   * @param <V> the type of elements in the set
   */
  public static <V> Set<V> remove(Set<V> s, V value) {
    var newSet = new HashSet<>(s);
    newSet.remove(value);
    return newSet;
  }

  /**
   * Adds an element to the head of a list
   *
   * @param l the list to add an element to
   * @param value the element to add
   * @return a new list with an element added to the head of the list
   * @param <V> the type of elements in the list
   */
  public static <V> List<V> add(List<V> l, V value) {
    return Stream.concat(Stream.of(value), l.stream()).toList();
  }

  public static <K, V> Map<K, V> put(Map<K, V> m, K key, V value) {
    var newMap = new HashMap<>(m);
    newMap.put(key, value);
    return newMap;
  }

  public static <K, V> Map<K, V> remove(Map<K, V> m, K key) {
    var newMap = new HashMap<>(m);
    newMap.remove(key);
    return newMap;
  }

  public static <A, B> Stream<SimpleEntry<A, B>> zip(List<A> as, List<B> bs) {
    return IntStream.range(0, Math.min(as.size(), bs.size()))
        .mapToObj(i -> new SimpleEntry<>(as.get(i), bs.get(i)));
  }

  public record Triple<A, B, C>(A first, B second, C third) {}

  public static <A, B, C> Stream<Triple<A, B, C>> zip(List<A> as, List<B> bs, List<C> cs) {
    return IntStream.range(0, Math.min(as.size(), Math.min(bs.size(), cs.size())))
        .mapToObj(i -> new Triple<>(as.get(i), bs.get(i), cs.get(i)));
  }

  public static <T> boolean hasDuplicates(Collection<T> collection) {
    return collection.stream().anyMatch(arg -> Collections.frequency(collection, arg) > 1);
  }

  public static <A, B> A foldRight(
      A initial, BiFunction<A, B, A> folder, Collection<B> collection) {
    var result = initial;
    for (var element : collection) {
      result = folder.apply(result, element);
    }
    return result;
  }
}

\end{lstlisting}